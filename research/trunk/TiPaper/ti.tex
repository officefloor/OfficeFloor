%%This is a very basic article template.
%%There is just one section and two subsections.
\documentclass{article}

%% Improve hyphenation
\hyphenation{op-tical net-works semi-conduc-tor}

\begin{document}

\title{Thread Injection and Continuation Injection}
\author{Daniel Sagenschneider}
\date{}
\maketitle

\abstract{}


\section{Introduction}

The proactor pattern \cite{proactor} is used as the basis of popular modern web
servers \footnote{TODO name servers as identified from the Netcraft 2012 survey}
to dispatch request handling (events) asynchronously to request handlers
(Asynchronous Operations).

The disadvantage of the Proactor pattern is that the ``Asynchronous Operation
Processor must be designed carefully to support prioritization and cancellation
of Asynchronous Operations'' \cite{proactor}.  When not relying on the Operation
System's asynchronous I/O, the Proactor Pattern does suggest using pooled
threads to execute the asynchronous operations.  Performance analysis of web
servers identifies the importance of tuning this correctly
\cite{tuning-important,tuning-os-important,low-server-footprint}.  The Thread
Injection Pattern presented in this paper focuses on addressing prioritization
of Asynchronous Operations by using multiple thread pools.

The Asynchronous Operation API has also become a single standard interface to
enable servicing dynamic web page content.  The discussion of the Proactor
Pattern focused on web servers providing static content with developers
implementing the Proactor Initiator.  Popular web servers however provide
dynamic web page content via developers implementing Asynchronous Operations to
service a request.  To enable portable implementations the Asynchronous
Operation interface has been standardised \footnote{CGI with for example PHP,
.NET TODO what is this?, JEE Servlet used by servers from the Netcraft 2012
survey}.

The nature of the Proactor Pattern is to enable using many Asynchronous
Operations in servicing a HTTP request.  For the popular Web Servers
standardising of the Asynchronous Operation interface however has focused on the
Proactor benefit of simplification of application synchronization by allowing
only one Asynchronous Operation to synchronously service the entire dynamic web
page content \footnote{JEE Servlet 3.x specification is acknowledging this by
introduction of the AsyncContext however requires the developer to be aware of
thread synchronization complexities}.  The Continuation Injection Pattern
presented in this paper focuses on addressing this by allowing multiple
Asynchronous Operations to be used for dynamic web page content. It also
provides the means for cancellation of Asynchronous Operations.

\section{Motivating Example}

TODO use Thread Injection tutorial as basis for example.
\begin{itemize}
  \item Thread Injection: talk about cache and database requests
  \item Continuation Injection: talk about splitting cache logic from database request 
\end{itemize}


\begin{table}[!t]
\renewcommand{\arraystretch}{1.3}
\caption{Example operations for servicing a HTTP request from database and caches. The dependencies of each operation are also listed.}
\label{tab:example_request_operations}
\centering
\begin{tabular}{l||l||l}
\hline
\bfseries Database Operations & \bfseries Cached Operations & \bfseries Dependencies \\
\hline\hline
Read data from socket & Read data from socket & Selector, Socket \\
\hline
Parse HTTP request & Parse HTTP request & Data read \\
\hline
Dispatch HTTP request & Dispatch HTTP request & HTTP request \\
\hline
Validate client data & Validate client data & HTTP request \\
\hline
Cache miss for data & Retrieve data from cache & Client data \\
\hline
Retrieve data from database & & Client data, \\
 & & Database connection \\
\hline
Render HTTP response & Render HTTP response & Database data \\
\hline
Write HTTP response & Write HTTP response & HTTP response, \\ 
 & & Socket \\
\hline
\end{tabular}
\end{table}


\section{Dependency Injection}

The naming of the Thread Injection and Continuation Injection pattern is because
of their close association to the Dependency Injection pattern.

The Dependency Injection pattern enables constructing objects using extrinsic
dependency information to load dependencies without the need of the client
requiring awareness of the dependencies \cite{ioc}.  This allows implementations
to be swapped without syntatic changes.

The Dependency Injection pattern therefore enables:
\begin{enumerate}
  \item A single interface for multiple implementations
  \item Meta-data about dependencies required for the implementation
  \item Means to provide one or more differing objects to the implementation
\end{enumerate}

These properties of the Dependency Injection pattern are important for the
patterns described in this paper:
\begin{itemize}
  \item Single interface is necessary for the Asynchronous Operations.
  \item Meta-data about dependencies provides Thread Injection the necessary information for dispatching Asynchronous Operations to the appropriate thread pool.
  \item Continuations encapsulating the Proactor Completion Handlers and Completion Dispatchers can be provided to Asynchronous Operations.
\end{itemize}

\section{Thread Injection}

The Thread Injection pattern addresses both the implementation of the
Asynchronous Operation Processor and providing an interface to provide
prioritization information between the Asychronous Operations and the
Asynchronous Operation Processor.

It's name is derrived from Asynchronous Operations having their executing thread
chosen (injected) for them by configuration similar to dependency injection.

As already mentioned, the Proactor pattern specifies that the Asynchronous
Operation API must be both portable and flexible.  While this applies to the API
between the Proactor Initiator and Asynchronous Operation, it also must apply to
the API between the Asynchronous Operation and Asynchronous Operation Processor
to decouple them.  This will enable portable developer implementations of
Asynchronous Operations (such as generating dynamic web page content). 

For the Thread Injection pattern, all Asynchronous Operations are constructed
via extrinsic dependency management \cite{ioc} allowing standard interfaces.
The construction of the Asynchronous Operation is encapsulated in the depenency
management factory.  This enables both the interfaces to the Proactor Initiator
and Asynchronous Operation Processor to be standardised.

Standardising interface between the Proactor Initiator and Asynchronous
Operation is not necessary for Thread Injection.  It is however necessary for
Continuation Injection discussed later in this paper.

The interface between the Asynchronous Operation and Asynchronous Operation
Processor is provided in Fig \ref{fig:AO_interface_AOP}.

\begin{figure}[!t]
\begin{verbatim}
    interface AsynchronousOperation {
    
        void execute();
        
        Class[] getRequiredDependencyTypes();
        
        void cancel(Exception cause);
    }
\end{verbatim}
\caption[Caption for Code]{Asynchronous Operation interface for the Asynchronous Operation Processor}
\label{fig:AO_interface_AOP}
\end{figure}

The \texttt{execute()} method is invoked by the thread of the chosen thread pool
to execute the implementation of the Asynchronous Operation.  The remaining
methods enable the prioritizing and cancelling of Asynchronous Operations.

\subsection{Prioritizing Asynchronous Operations}

The prioritization provided by the Thread Injection pattern is achieved by using
the \texttt{getRequiredDependencyTypes()} method to obtain the extrinsic
depenency management information \footnote{Dependency Injection frameworks using qualification to identify dependencies of the same type may return a type object containing both class and qualifier rather than just a class object.}.

The developer configures one or more thread pools responsible for Asynchronous
Operations with a particular type of dependency. Table
\ref{tab:example_request_thread_pools} provides an example set of thread pools
for the motivating example.

\begin{table}[!t]
\renewcommand{\arraystretch}{1.3}
\caption{Example assigning of thread pool to Asynchronous Operations by dependency}
\label{tab:example_request_thread_pools}
\centering
\begin{tabular}{l||l||l}
\hline
\bfseries Thread Pool & \bfseries Dependency & \bfseries Asynchronous Operation \\
\hline\hline
Network & Selector & Read data from socket \\
\hline
Database & Database connection & Retrieve data from database \\
\hline
Default & - & Parse HTTP request, \\
& & Dispatch HTTP request, \\
& & Validate client data, \\ 
& & Render HTTP response, \\
& & Write HTTP response \\
\hline
\end{tabular}
\end{table}

As the \texttt{Retrieve data from database} operation is assigned to be executed
by its own thread pool, the Thread Injection pattern allows the remaining
operations to be undertaken.  This is because only the threads of the
\texttt{Database} thread pool may undertake the operations which incur database
blocking I/O.  The remaining operations have their own thread pools allowing
them to continue to be executed even if the database I/O is causing thread
starvation within its own thread pool.

The Thread Injection pattern can therefore be considered a style of cohort
scheduling \cite{cohort} to group operations with similar dependencies (and
inferring from that similar functionality).  It however works at the application
scheduling level and allows use of any Operating System thread scheduling
algorithms.  Furthermore, tuning the thread pools (such as restricting number of
threads or changing the pool's thread nice values) allows prioritizing threads and
subsequently Asynchronous Operations.

As the dependencies for each Asynchronous Operation is static, at application
start up time the Asynchronous Operation Processor may also preprocess the
mapping of Asynchronous Operation to Thread Pool to reduce runtime decision
overheads.  This pre-mapping of Asynchronous Operations may also provide
warnings where dependencies of an Asynchronous Operation may make it possible to
be mapped to multiple thread pools.  Different conflict mapping resolutions may
be employed, however ordering the thread pools and assigning based on first
match is a sufficiently simple and understandable algorithm.

The result is that the Asynchronous Operations for servicing the cached request
can now be prioritized even if the database is causing significant delays that
would other wise have caused thread starvation when only a single thread pool
for all Asynchronous Operations was employed.

From the perspective of the Asynchronous Operation, the developer is now able to
select which thread pool will execute each Asynchronous Operation.  This
effectively injects the thread for execution similar to injecting a dependency
for use.  Hence the name of the pattern, Thread Injection.


\subsection{Cancelling Asynchronous Operations}

The \texttt{cancel(Exception)} provides the means for the Asynchronous Operation
Processor to cancel the Asynchronous Operation.  Once invoked, the Asynchronous
Operation Processor may discard the Asynchronous Operation.

The implementaion of the \texttt{cancel(Exception)} is the responsibility of the
Asynchronous Operation as it will likely be specific to the operation's
implementation.  The Continuation Injection pattern explained later provides a
means to alleviate this from the Asynchronous Operation implementation.

As each thread pool is executing Asynchronous Operations for a particular
dependency (or set of dependencies), it allows for stage management and
admission control regarding the dependency \cite{seda}.  Both the number of
threads and dependencies may be dynamically altered to improve throughput. 
However when maximum throughput is reached, additional Asynchronous Operations
above this threshold can be cancelled.


\section{Continuation Injection}

The implementation of the Completion Dispatcher within this pattern is where
Continuation Injection may occur.

Break request into \cite{pipeline} of tasks that are connected together via
continuations \cite{continuations} that encapsulate triggering an event and
dispatching for another event handler.

The continuations are also extended beyond the web server via URL continuations
\cite{url-continuation}.


\section{Implicit Thread}

When using the Thread Injection and Continuation Injection pattern together,
implicit threads may be used to reduce thread context switching.

The continuation may borrow the thread of the invoking Asynchronous Operation if
it results in execution by the same thread pool.  The Proactor Pattern
stipulates that Asynchronous Operations ``are performed without borrowing the
application's thread of control'' which is the focus of the Reactor Pattern
\cite{reactor}.  Rather than dispatching back to the same thread pool the
continued Asynchronous Operations may safely borrow the Asynchronous Operation
invoking thread to avoid the overhead of a thread context switch.

This same idea is also extended to the default thread pool for the Thread
Injection Pattern.  For an Asynchronous Operation not requiring any dependencies
for dispatching to a specific thread pool, the Asynchronous Operation is deemed
cheaper to be executed by the current Asynchronous Operation's thread than
incurring the cost of a thread context switch.

The borrowed thread is the implicit thread.  Like an implicit continuation that
executes the next operation \cite{continuations}, the implicit thread executes
the next operation unless a explicit thread is required.  This can result in the
Web Server servicing the entire request without a thread context switch should
no explicit threads be required for servicing the request (e.g. web page content
obtained from cache).


\section{Drawbacks}

\emph{Hard to debug}: The Thread Injection and Continuation Injection Patterns
have not resolved the Proactor issue of being hard to debug.  Furthermore,
dependencies that are not appropriately thread safe can create very difficult to
identify bugs.

\section{Known Uses}

OfficeFloor is the only known use of the Thread Injection and Continuation
Injection pattern.  It is hoped that by providing more awareness of these
patterns that adoption of these patterns will increase.


\bibliographystyle{style/IEEEtran}
\bibliography{ti}

\end{document}
