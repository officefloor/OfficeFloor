%% OfficeFloor - http://www.officefloor.net
%% Copyright (C) 2013 Daniel Sagenschneider
%%
%% This program is free software: you can redistribute it and/or modify
%% it under the terms of the GNU General Public License as published by
%% the Free Software Foundation, either version 3 of the License, or
%% (at your option) any later version.
%%
%% This program is distributed in the hope that it will be useful,
%% but WITHOUT ANY WARRANTY; without even the implied warranty of
%% MERCHANTABILITY or FITNESS FOR A PARTICULAR PURPOSE.  See the
%% GNU General Public License for more details.
%%
%% You should have received a copy of the GNU General Public License
%% along with this program.  If not, see <http://www.gnu.org/licenses/>.
%%
%% While this document is not a program, it conveys the underlying design 
%% of OfficeFloor (it is the expression of how to implement the ideas of 
%% Thread Injection, Implicit Thread, Continuation Injection, Operation 
%% Orchestration, Inversion of Control) and as such any program derived from 
%% the contents (expression) of this document is considered conveying 
%% (copying/modifying) the OfficeFloor expression and is therefore subject 
%% to the licensing of OfficeFloor.


%%This is a very basic article template.
%%There is just one section and two subsections.
\documentclass[prodmode]{style/acmlarge}

% Metadata Information
\acmVolume{V}
\acmNumber{N}
\acmArticle{A}
\articleSeq{S}
\acmYear{YYYY}
\acmMonth{0}

% Package to generate and customize Algorithm as per ACM style
\usepackage[ruled]{style/algorithm2e}
\SetAlFnt{\algofont}
\SetAlCapFnt{\algofont}
\SetAlCapNameFnt{\algofont}
\SetAlCapHSkip{0pt}
\IncMargin{-\parindent}
\renewcommand{\algorithmcfname}{ALGORITHM}

% Page heads
\markboth{D. Sagenschneider}{Thread Injection and Continuation Injection}


\title{Thread Injection and Continuation Injection}
\author{DANIEL SAGENSCHNEIDER \affil{daniel@officefloor.net}}

\begin{abstract}
The Thread Injection and Continuation Injection patterns provide implementing
algorithms for the participants of the Proactor pattern to enable inverting
control for the thread executing and the ordering of operations.  These patterns
combined with Dependency Injection provide an Inversion of Control pattern.
\end{abstract}

\category{X.Y.Z}{To}{be}[determined]

\terms{Design, Performance, Standardization}
\keywords{Continuation Injection, Implicit Thread, Operation Orchestration, Thread Injection}

\acmformat{Sagenschneider, D. 2013.Thread Injection and Continuation Injection.}

\copyr{Copyright 2013 is held by the author}

\begin{document}

\begin{bottomstuff}
This work is the result of the author's development of OfficeFloor.\\
Author's address: D. Sagenschneider, at home; email: daniel@officefloor.net\\

Permission to make digital or hard copies of all or part of this work for
personal or classroom use is granted without fee provided that copies are not
made or distributed for profit or commercial advantage and that copies bear this
notice and the full citation on the first page. To copy otherwise, to republish,
to post on servers or to redistribute to lists, requires prior specific
permission. A preliminary version of this paper was presented in a writers'
workshop at the 18th European Conference on Pattern Languages of Programs
(PLoP).
\end{bottomstuff}

\maketitle


\section{Introduction}

%% TODO: use \textsc{..} for names of patterns.  Also make statement in overview about small caps.

The \textsc{thread-per-request} pattern \cite{thread-per-request} (that can be
considered a refinement of the \textsc{synchronous multi-threaded} pattern
\cite{proactor}) is used as the basis of popular modern web
servers\footnote{Apache, Microsoft IIS, Nginx and JEE identified from the
Netcraft November 2012 survey.  Google Web Server is also identified as
popular.} to service requests concurrently.

Many modern Web Servers improved their concurrency of servicing high loads of
static web content by using the \textsc{proactor} pattern \cite{proactor} to
enable use of asynchronous I/O operations.  The \textsc{thread-per-request}
pattern requires a thread to be allocated to each request to sequentially and
synchronously execute all operations for servicing a request.  This includes I/O
operations that tie up thread resources on the Web Server requiring more threads
and reducing the concurrency of the Web Server.  The \textsc{procator} pattern
overcomes this issue by executing operations asynchronously not requiring the
Web Server threads to block (e.g. use of Operating System asynchronous I/O
operations).

However, Web Servers continue to use the \textsc{thread-per-request}
pattern\footnote{CGI/FastCGI with for example PHP scripts, Microsoft's
HTTP.sys/WAS and JEE Servlets.} for servicing dynamic web content.  Using the
\textsc{thread-per-request} pattern allows for more intuitive developer
implementations of dynamic web content servicing at the cost of decreased
concurrency for the Web Server.

Due to demands on Web Servers to resolve such issues as the Reverse 10K problem
\cite{reverse-ten-k-problem}, the \textsc{thread-per-request} pattern is
requiring developers to manually manage the threading issues from requiring
increased concurrency.  Should the developer want to use asynchronous I/O
operations for the increased concurrency, the developer is required to block the
request's thread of control, undertake the concurrent operations and continue
the request's thread of control when all concurrent operations
complete\footnote{JEE Servlet 3.x AsyncContext is allowing asynchronous
operations but requires the developer to still be involved in thread
synchronization complexities.}.

To use the \textsc{proactor} pattern to service dynamic web content causes its
drawback of scheduling and controlling outstanding operations \cite[p.
8]{proactor} to require resolution.  Servicing static web content follows a
standard sequence of operations that enables providing a bespoke solution to
schedule and control outstanding operations.  In contrast, servicing dynamic web
content does not follow a standard sequence of operations and therefore requires
a more generic solution that is ``designed carefully to support prioritization
and cancellation'' \cite[p. 8]{proactor} of the operations.

The Thread Injection and Continuation Injection patterns presented in this paper
provide a generic solution to scheduling/controlling operations and enables use
of asynchronous operations for increased concurrency in servicing dynamic web
content.  Furthermore, using the Thread Injection and Continuation Injection
patterns with the Dependency Injection pattern \cite{ioc} provides the
decoupling necessary for an Inversion of Control pattern.


\section{Overview}

This paper first presents a motivating example for both the Thread Injection
pattern and Continuation Injection pattern.  The Continuation Injection pattern
and Thread Injection pattern are then presented.  After presenting the patterns
individually, the Thread Injection and Continuation Injection patterns are used
with the Proactor and Dependency Injection patterns to realise an Inversion of
Control pattern.  The Inversion of Control pattern provides resolution of the
motivating example.  Discussion is then provided on using the patterns to
implement a Web Server.

For ease of reading, figure \ref{fig:InjectionInterfaces} describes the combined
interfaces for the Thread Injection and the Continuation Injection patterns.
Also for completeness of the Inversion of Control pattern presented in this
paper, the well documented Dependency Injection pattern \cite{ioc} interface is
also included via the \texttt{DependencyInjectionFactory}\footnote{A rudimentary
example implementation of the \texttt{DependencyInjectionFactory} is wrapping
Spring's \cite{spring} \texttt{DefaultListableBeanFactory} to implement
\texttt{createOperation(\ldots)} by calling \texttt{getBean(\ldots)} and
implement \texttt{getRequiredDependencyTypes(\ldots)} by recursively calling
\texttt{getBeanDefinition(\ldots)}/\texttt{getDependsOn()} to create the list of
types.  A more functional example implementation would utilize Spring's
\texttt{ApplicationContext} to manage the context for the Asynchronous
Operations and their dependencies.} and the \texttt{getDependency(\ldots)}
method\footnote{The \texttt{getDependency(\ldots)} method provides retrieval of
the extrinsically defined dependencies for the Asynchronous Operation.  An
example implementation of this method would be calling Spring's \cite{spring}
\texttt{getBean(\ldots)} method.}.

%% TODO: fix up Asynchronous Operation to be component

\begin{figure}[tp]
\centering
\begin{verbatim}
    interface Component {
        void execute(ComponentContext context);
        void cancel(Exception cause);
        String[] getContinuationIds();
    }

    interface ComponentContext {
        Object getDependency(String dependencyId);
        Future doContinuation(String continuationId, Object parameter);
        void continueComponent();
    }
    
    interface DependencyInjectionFactory {
        Type[] getRequiredDependencyTypes(String componentId);
        Component createComponent(String componentId, Object parameter);
    }
\end{verbatim}
\caption{Combined Thread Injection, Continuation Injection and Dependency Injection pattern interface\footnotemark}
\label{fig:InjectionInterfaces}
\end{figure}
\footnotetext{Integer identifiers may be used for fast array look ups rather than strings.}

As the Thread Injection and Continuation Injection patterns provide implementing
algorithms for the Proactor pattern participants, the Proactor pattern
participants have been summarised in table \ref{tab:ProactorParticipants}.

\begin{table}[t]
\tbl{Proactor pattern participants.}{%
\begin{tabular}{|l|l|}
\hline
\bfseries Participant & \bfseries Responsibilities \\
\hline
Proactor Initiator & Initiates the Asynchronous Operation and \\ 
 & registers both a Completion Dispatcher and \\
 & Completion Handler with the Asynchronous \\
 & Operation Processor.\\
\hline
Completion Dispatcher and & Notified of the completion and handles the \\
Completion Handler & completion of an Asynchronous Operation. \\
\hline
Asynchronous Operation & Operation to be undertaken. \\
\hline
Asynchronous Operation Processor & Execution of the Asynchronous Operations. \\
\hline
\end{tabular}}
\label{tab:ProactorParticipants}
\end{table}



\section{A Motivating Example}

A motivating example for the Thread Injection and Continuation Injection
patterns is servicing dynamic web content.  Servicing a typical request for
dynamic web content may use both cached/static content and data retrieved from a
database.

Due to the standardizing of the Asynchronous Operation interface by popular
modern Web Servers (i.e thread-per-request pattern), requests involving no I/O
or heavy I/O are serviced by the same pool of threads.  If for example the
database I/O becomes slow causing all threads to block, requests requiring only
cached content are not prioritized as they are starved of a thread.

Furthermore, for requests requiring only in memory cached content, popular
modern Web Servers continue to use a separate thread that incurs the cost of one
or more thread-context switches.  If the Web Server knew further details about
the nature of each Asynchronous Operation, it should allow the Asynchronous
Operation to borrow the thread to reduce thread-context switching.

Table \ref{tab:example_request_operations} lists the operations for servicing an
example HTTP request.  The operations executed differ based on whether there is
a cache hit or miss.  Ideally, as explained above, all operations up to and
including the \texttt{Retrieve data from cache} should be borrowing the thread
to avoid a thread-context switch.  A separate thread should then only be used if
the \texttt{Retrieve data from database} operation needs to be executed so that
the main thread does not block causing potential for thread starvation.  The
remaining operations should then again be borrowing the thread to avoid
thread-context switching.

\begin{table}[t]
\tbl{Example operations for servicing a HTTP request from database and cache. The dependencies of each operation are also listed.}{%
\begin{tabular}{|l|l|l|}
\hline
\bfseries Cache Miss Operations & \bfseries Cache Hit Operations & \bfseries Dependencies \\
\hline
Read data from socket & Read data from socket & Selector, Socket \\
\hline
Parse HTTP request & Parse HTTP request & Data read \\
\hline
Dispatch HTTP request & Dispatch HTTP request & HTTP request \\
\hline
Validate client data & Validate client data & HTTP request \\
\hline
Retrieve data from cache & Retrieve data from cache & Client data, \\
(cache miss) & (cache hit) & Cache \\
\hline
Retrieve data from database & - & Client data, \\
 & & Database connection \\
\hline
Render HTTP response & Render HTTP response & Database data \\
\hline
Write HTTP response & Write HTTP response & HTTP response, \\ 
 & & Socket \\
\hline
\end{tabular}}
\label{tab:example_request_operations}
\end{table}



\section{Continuation Injection}


\subsection{Problem}

\textbf{How to loosely couple the collaboration of components?}  Components
provide discrete pieces of functionality (application behaviour) that may be
developed by different developers.  To ensure greater re-usability of a
component, how can the invocation of that component be loosely coupled to enable
re-use in more contexts?


\subsection{Context}

Both the \textsc{thread-per-request} pattern \cite{thread-per-request} (basis of
many mainstream Web Servers\footnote{For popular Web Servers (Netcraft November
2012 survey) dynamic web content is serviced by CGI/FastCGI with for example PHP
scripts, Microsoft's HTTP.sys/WAS and JEE Servlets.}) and the \textsc{proactor}
pattern \cite{proactor} (basis of event-driven Web Servers) impose tight
coupling on invocation of components.  The \textsc{thread-per-request} pattern
enables invoking components by synchronous methods that is intuitive for
developers \cite[p. 2]{proactor}.  In contrast, the \textsc{proactor} pattern
enables asynchronously invoking a constructed component by registering it for
execution by another thread of control (allowing to execute components
concurrently).  In both patterns, they tightly couple the invocation as the
\textsc{thread-per-request} pattern must have the invoker provide the thread of
control and the \textsc{proactor} pattern must have the invoker construct the
invoked component.

Loosely coupling the invocation of components leads to better re-use of
components.  ``Tight coupling leads to monolithic systems, where you can't
change or remove a \ldots [component] without understanding and changing other
\ldots [components]'' \cite[p. 24-25]{gof}.  Having to change components to
invoke them negates the ability to re-use components.


\subsection{Forces}

The \textsc{thread-per-request} pattern imposes a synchronous interface that
prevents asynchronous use of other components.  As the number of downstream
systems increases from a typical single database to a variety of services (e.g.
Reverse 10K problem \cite{reverse-ten-k-problem}), the
\textsc{thread-per-request} pattern requires the developer to manually handle
the multi-threading issues of concurrent communication to downstream systems to
efficiently service a request.  It is therefore better to use the
\textsc{proactor} pattern in this circumstance.

The \textsc{proactor} pattern requires increased coding by the developer to
invoke a component.  To invoke a component, the developer must construct the
component, register it for execution and then handle completion of the
component.  Furthermore, handling completion of multiple concurrently executing
components may not be intuitive for many developers.  When concurrency is not
necessary, using methods to synchronously invoke components via the
\textsc{thread-per-request} pattern can produce less code that will likely be
more intuitive for developers.  Therefore, invoking components should be via a
method invocation for potentially less code and easier understanding of the
code.

The synchronous nature of the \textsc{thread-per-request} pattern also imposes
constraints in inverting the control over the order of component execution.  As
the sequence of components executed is undertaken via synchronous direct method
calls, the \textsc{thread-per-request} pattern provides no indirection to invert
the control over the order components are executed.

The \textsc{proactor} pattern also imposes hard-coded order over the components
invoked (albeit they may end up being executed concurrently).  The invoker must
construct the component to be asynchronously invoked and therefore this does not
provide the indirection to control invoking a different component.

The \textsc{thread-per-request} pattern and \textsc{proactor} pattern further
tightly couple invoking a component by the invoker being required to handle
possible exceptions.  The invoker may not be appropriately responsible to handle
a resulting exception and this should be delegated to another component.

For a framework (e.g. Web Server) to extend itself by invoking components, the
framework predefines the component's interface to enable invoking it.  However,
components should be able to invoke other components for re-use.  To enable
re-use of these components the interface to invoke components should be
standardised.


\subsection{Solution}

%% TODO: useful to have some kind of visualization of the main solution in the solution part, a sequence diagram example or some other illustration

\textbf{Provide a context object to the component to enable indirectly invoking other components.}

The Continuation Injection pattern relies on providing the
\texttt{ComponentContext} (Fig \ref{fig:InjectionInterfaces}) to the component.
The \texttt{ComponentContext} contains a mapping of \texttt{continuationId} to a
handling component.  As each mapping is in effect a continuation (''goto'' for
executing the component), this allows injecting the necessary continuations into
the component.

The Continuation Injection pattern provides algorithms implementing the
\textsc{proactor} pattern's Proactive Initiator, Asynchronous Operation and
Completion Dispatcher/Handler participants.  The continuation triggers the
Proactor Initiator to register an Asynchronous Operation containing the desired
component for execution.  The also registered Completion Dispatcher/Handler
notifies the Proactor Initiator when the Asynchronous Operation and subsequently
the component completes.  Using a Proactor Initiator per request to track
completion of Asynchronous Operations (components) allows providing a response
once all components for the request are complete.

Components are constructed via extrinsic \textsc{dependency injection}
\cite{ioc}. The \textsc{dependency injection} context, that manages the
life-cycle of dependencies, may be aligned to the Proactor Initiator life-cycle.
 This allows dependencies for the components to be specific to the request being
serviced.  It also allows the state/resources within these dependencies to be
managed within the life-cycle of servicing the request.

Re-using dependencies between components allows sharing state between
components.  For the \textsc{thread-per-request} pattern method signatures
tightly couple sharing state by requiring particular parameters and only
providing a single return value.  For the \textsc{proactor} pattern, the invoker
must have all state available to construct the Asynchronous Operation and
Completion Dispatcher/Handler which tightly couples the invocation.  Providing
mutable state within dependencies that are re-used across components allows
components to share state.  This is similar to a \textsc{thread-per-request} Web
Server that shares state between components by attributes (dependencies) within
request, session and application scopes.  The invoking component only needs the
dependencies (state) it requires and does not need to provide the entire set of
dependencies (state) for the invoked component\footnote{To further decouple the
state sharing for the component, the \texttt{ComponentContext} may contain
developer configuration to map the \texttt{dependencyId} (Fig
\ref{fig:InjectionInterfaces}) to the context specific identifier for the
dependency. This allows components to reference the same dependency by their own
identifier.}.

The \texttt{doContinuation(\ldots)} method (Fig \ref{fig:InjectionInterfaces})
allows the component to invoke another component.  As the
\texttt{continuationId}s are static for each component, developer configuration
provides the continuation mapping to the handling component.  The method does
allow one argument to be passed to the invoked component.  This is for more
intuitive development by associating the invocation with passing state.  If
multiple arguments are required, they should be encapsulated into an object for
passing.  To maintain loose coupling, the invoked component obtains the argument
as an injected dependency so it need not differentiate between dependencies and
parameters.  Figure \ref{fig:DoContinuationSequenceDiagram} provides a sequence
diagram of invoking a component via the \texttt{doContinuation(\ldots)} method.

\textbf{TODO: provide sequence diagram for the above}

As the \texttt{doContinuation(\ldots)} method asynchronously invokes the
component, there is no return value from the invoked component.  Return values
are achieved by mutating state of a dependency and having the invoking component
re-use the dependency on completion of the continuation \texttt{Future}.  This
further decouples the invoking component, as it may obtain multiple return
values via multiple dependencies.

Exceptions are handled by being mapped to a \texttt{continuationId}.  The
Asynchronous Operation catches the component's exception and invokes the
\texttt{doContinuation(\ldots)} method with the appropriate
\texttt{continuationId} for the exception type.  The exception is used as the
continuation argument to pass it to the handling component.  The invoking
component is decoupled from handling exceptions from the invoked component as
the exception is handled by another component.

Developer tools may use the \texttt{getContinuationIds()} method to obtain the
list of possible continuations from a component.  This will enable meta-data
driven tools to aid the developer in providing the continuation mapping
configuration.

The \texttt{continueOperation()} method (Fig \ref{fig:InjectionInterfaces})
enables executing the current component again.  Frameworks managing the
components may be implemented to defer execution of a component again until
invoked continuations by the component are completed.  This for example allows
deferring the execution of the component until an invoked Operating System
asynchronous I/O operation completes - a key aspect of the \textsc{Proactor}
pattern for improved concurrency.

The intuitiveness of sequential programming is maintained by using implicit
continuations \cite{continuations} to sequentially chain components.  Each
\texttt{ComponentContext} also maintains an optional implicit continuation. 
Should no continuation be invoked (i.e. \texttt{doContination(\ldots)} method
invoked or an exception thrown) the implicit continuation is invoked on
completion of the component.  This allows chaining together components to be
executed sequentially.

The continuation mapping configuration may also have qualifying attributes that
isolate components to particular \textsc{dependency injection} contexts.  The
continuation may be configured to invoke the component (and its subsequent
invoked components) within another \textsc{dependency injection} context to not
share state.

To share state between the isolated \textsc{dependency injection} contexts, the
isolated contexts may be created inside a shared context.  Each dependency type
is bound by developer configuration to be within the isolated context or the
shared context.  The instance of a dependency bound to an isolated context may
only be used by component instances within the same isolated context instance. 
However, the instance of a dependency bound to the shared context is re-usable
across all component instances\footnote{OfficeFloor \cite{officefloor} mimics
the thread to process relationship by creating isolated ''thread'' contexts
within a shared ''process'' context.}.  Figure
\ref{fig:DependencyInjectionContexts} shows the scope for dependencies within
isolated and shared context instances.

\textbf{TODO: provide graph of components showing implicit, thread, exception}


\subsection{Example}

The \texttt{Retrieve data from cache} operation (Table
\ref{tab:example_request_operations}) from the motivating example will be used
as an example of Continuation Injection.  Figure
\ref{fig:Example_Method_Operation} shows the example developer implementation
code.  A generic component implementation is used to reflectively invoke the
\texttt{retrieveData(\ldots)} method. It will:
\begin{enumerate}
  \item Obtain an instance of the \texttt{CacheOperation} via its context's \texttt{getDependency(\ldots)} method.
  \item Obtain both the \texttt{key}\footnote{\texttt{key} is a continuation argument from the previous component.} and \texttt{cache} again via the \texttt{getDependency(\ldots)} method.
  \item Instantiate a proxy implementation of the \texttt{CacheContinuation} interface that implements the \texttt{cacheMiss(\ldots)} method by invoking the \texttt{doContinuation(\ldots)} method. 
  \item Reflectively invokes the \texttt{retrieveData(\ldots)} method with the above arguments.
\end{enumerate}

\begin{figure}[tp]
\centering
\begin{verbatim}
  interface CacheContinuations {
    void cacheMiss(String key);
  }

  class CacheOperation {    
    public Data retrieveData(String key, Cache cache, CacheContinuations continuations) throws IOException {
        Data data = cache.get(key);
        if (data == null) {
            continuations.cacheMiss(key);
            return null; // finish operation
        }
        return data;
    }
  }
\end{verbatim}
\caption{Example developer implementation code for a component\footnotemark}
\label{fig:Example_Method_Operation}
\end{figure}
\footnotetext{\texttt{retrieveData} may also be a function should the implementing programming language support functions.}

Continuations from the \texttt{retrieveData(\ldots)} method are:
\begin{itemize}
  \item \texttt{cacheMiss(\ldots)} which will be mapped to \texttt{Retrieve data from database}.
  \item Implicit continuation which is mapped to the \texttt{Render HTTP response}\footnote{The return value from the method is used as the continuation argument.}.
  \item \texttt{IOException} that is mapped to a component providing an error message page.  It may also be mapped to \texttt{Retrieve data from database} to attempt to continue servicing the request.
\end{itemize}


\subsection{Consequences}

As the continuation asynchronously invokes the component, the returned
\texttt{Future} may be used to determine when the potential three of invoked
components has completed.  The \texttt{Future} would utilise the Proactor
Initiator's tracking of Asynchronous Operation completion to determine when the
tree of Asynchronous Operations realising the continuation has completed.  This
enables process continuations \cite{process-continuation} to be spawned by
repeatedly calling \texttt{doContinuation(\ldots)}.  The process continuations
aid in resolving the reverse 10K problem \cite{reverse-ten-k-problem} by
executing multiple components concurrently.

\textbf{TODO: combine above and below paragraph}

Having the components, in this isolated context, execute via implicit
continuations is similar to spawning a thread.  The continuation results in two
sequential chains of components being executed that do not share state.  This
allows the intuitiveness of the \textsc{synchronous multi-threaded} pattern
\cite{proactor} for handling concurrency.

As continuations asynchronously invoke components, they need not only be invoked
by components. Figure \ref{fig:DC_interface} illustrates a possible interface
that can be injected into dependencies to invoke a component\footnote{No
\texttt{continutionId} is necessary as may inject multiple instances
representing the different continuations.}.  This enables dependencies to
effectively become Active Objects \cite{active-object} when
necessary\footnote{The Web Server HTTP(S) socket dependency would for example
typically be implemented by invoking an injected continuation to service
requests.}.

\begin{figure}[tp]
\begin{verbatim}
    interface DependencyContinuation {
        void doContinuation(Object parameter);
    }
\end{verbatim}
\caption{Injected interface for a dependency to invoke a continuation (Active Object)}
\label{fig:DC_interface}
\end{figure}

While the \textsc{proactor} pattern focuses on asynchronous interaction between
Asynchronous Operations, the \texttt{ComponentContext} does not restrict the
servicing of the continuation invocation from being synchronous.  In cases where
it is more efficient to synchronously invoke the component and avoid the
thread-context switching overheads, this can be controlled by the developer with
a flag in the continuation mapping configuration.  In contrast, having
synchronous interaction be imposed (as per the \textsc{thread-per-request}
pattern) causes overheads for developers to manually manage in code the use of
asynchronous interaction.

The mapping of \texttt{continuationId} to \texttt{componentId} (component) is
contained within configuration.  This enables changing the handling component
without changing code\footnote{The continuation argument may be type validated
against the handling component to reduce runtime errors and therefore may
restrict some mapping changes.  An adapting component may however be mapped in
to remove this restriction by transforming the argument to the necessary type
for the handling component.}.  Changing the handling components enables
re-ordering the sequence of components to service a request.

To ease debugging and improve developer understanding of the application, the
mapping configuration of \texttt{continuationId}  to \texttt{componentId} may be
graphical.  The components are represented as nodes with the continuation
mappings being directed lines between these nodes.  Due to the similarity of
configuration with service orchestration, this is identified as Operation
Orchestration\footnote{Operation orchestration may also be identified as
function, procedure or method orchestration depending on the implementing
language.}.

%% TODO: negative consequences of the pattern?  (real negative is all collaboration is asynchronous meaning there is no return values.  This follows sequence of operations in flow of execution but may create coding problems when a return value is required.  Note Inversion of Control pattern resolves this issue by combining Continuation Injection with Dependency Injection.)
\textbf{TODO negative: thread of control}
\textbf{TODO negative: indirection between components may make hard to see flow from code (Operation Orchestration resolves this)}
\textbf{TODO negative: can not be used without framework.}
\textbf{TODO negative: sharing state is via side effect (change dependency state).  This is however is a similar development model to web development where state is shared as attributes in request, session, and application scopes.}
\textbf{TODO negative: hard to debug for proactor pattern. Not sequential but no handler required.}
\textbf{TODO negative: prioritization and cancellation for proactor pattern not resolved. Part of related patterns.}
\textbf{TODO negative: stack trace from sequential programming is not available but may compensate by keeping track of the tree of executed components.}


\subsection{Related Patterns}

The \textsc{dependency injection} pattern \cite{ioc} is necessary to enable
managing state between components\footnote{As the components contain the
functionality (application behaviour), the \textsc{dependency injection} pattern
in the context of the relationship to the Continuation Injection pattern) may
better be described as the \textsc{state injection} pattern}.

The Continuation Injection pattern provides an implementing algorithm for the
{proactor} pattern's Proactor Initiator, Asynchronous Operation, Completion
Dispatcher/Handler participants.

Thread Injection explained later in this paper provides a means for
prioritization and cancellation of the Asynchronous Operations required for
implementing the remaining Asynchronous Operation Process participant of the
\textsc{proactor} pattern.


\subsection{Known Usage}

Continuation Injection is an implementation of the Actor Model \cite{actors} as
it adheres to the principles of the Actor Model.  The component is the actor.
The asynchronous communication between components decouples the continuation
argument (message) from the sending component (actor).  The \texttt{componentId}
provides an address for a component (actor).  The provided
\texttt{continuationId}s restricts the components (actors) that may be used.

Systems integrated with a queuing infrastructure are heavy weight
implementations of the Continuation Injection pattern.  The continuation is the
queue's client interface with the queue infrastructure providing the necessary
indirection to map the handling component of a message (continuation argument).

Continuation Injection can also be considered a form of continuation-passing
style \cite{continuations} except that the continuation is injected rather than
passed as a parameter.  The added benefit of injecting the continuation is that
the component is free to depend on as many continuations as necessary.  It also
means that as the application evolves the component encapsulates potential
changes requiring different continuations.  The indirection provided by the
Continuation Injection pattern allows managing these changes as configuration
changes rather than code changes.

Subsequently event-driven architectures are an implementation of the
Continuation Injection pattern.  A component runs in a loop waiting for
events\footnote{The main loop component can be avoided by mapping events
directly to handling components.}.  On receiving an event, the component invokes
its appropriate continuations to have the event handled by other components. 
Note that event-driven architectures are a more rigid implementation as
components are in most cases only synchronously invoked.

OfficeFloor \cite{officefloor} implements the Continuation Injection pattern and
will be discussed later in this paper as it also implements the Thread Injection
pattern to provide an Inversion of Control implementation.



\section{Thread Injection}

%% TODO: is this pattern like a refinement to Proactor pattern?  (last sentence bothered me a little bit)
%% TODO: So at least in case of this pattern, the problem statement should be after the context part or rephrased as it is not very standalone statement as the reader doesn't know what algorithm should be provided.

\subsection{Problem}

\textbf{Provide an algorithm for prioritization and cancellation of Asynchronous Operations.}


\subsection{Context}

The disadvantage of the Proactor pattern is that the ``Asynchronous Operation
Processor must be designed carefully to support prioritization and cancellation
of Asynchronous Operations'' \cite[p. 8]{proactor}.  Performance analysis of web
servers identifies the importance of this prioritization tuning
\cite{tuning-important,low-server-footprint,tuning-os-important}. The Proactor
pattern, however, does not provide algorithms for implementing its participants
and leaves the prioritization and cancellation of Asynchronous Operations to the
developer.  It does suggest, when not relying on the Operating System's
asynchronous I/O, to use a pool of threads to execute the Asynchronous
Operations but provides no further detail on implementing the pool of threads.


\subsection{Forces}

Tuning the prioritization is important to the performance of a Web Server.
Serving static web content follows a similar sequence of operations making the
use of a single thread pool sufficient.  In contrast, servicing dynamic content
has significantly varying operations.  Utilising a single thread pool to service
all dynamic content operations can result in tuning trade-offs that may result
in performance degradation of the Web Server during particular load profiles
(e.g. high load of requests for database data that are starving requests for
cached data of a thread).  Better isolation of operations is necessary to reduce
trade-offs in performance tuning.

Having an ability to cancel operations is important to avoid overloading the Web
Server.  As load increases on the Web Server, the resources available to the Web
Server will be exhausted causing delays in servicing requests and in some
scenarios cause the Web Server to crash (e.g. exhausting available memory). 
Operations must be able to be cancelled to reduce the load on the Web Server. 
However, only operations causing bottlenecks should be cancelled (e.g. a request
for cached data should not be cancelled if it is the database that is
overloaded).

Both prioritization and cancellation of operations is specific to the
application behaviour.  To enable portable implementations of Asynchronous
Operations, a standard interface is necessary that provides application specific
information regarding prioritization and cancellation of Asynchronous
Operations.


\subsection{Solution}

%% TODO: problem and solution should match
%% TODO: add an illustration of the solution

\textbf{Use multiple thread pools and based on meta-data of an Asynchronous Operation assign it to a particular thread pool.}

All Asynchronous Operations are constructed via extrinsic dependency management
\cite{ioc}.  The construction of the Asynchronous Operation is encapsulated in
the dependency injection factory and as such the Asynchronous Operation
Processor is provided a standard interface for all operations (Fig
\ref{fig:InjectionInterfaces}).  This enables for portable implementations of
Asynchronous Operations by developers.

The prioritization is achieved by the
\texttt{getRequiredDependencyTypes(\ldots)} method providing the extrinsic
dependency meta-data for an Asynchronous Operation\footnote{Dependency Injection
frameworks using qualification to identify dependencies of the same type may
return a type object containing both class and qualifier rather than just a
class.  The thread pool matching may then incorporate the qualifier.}.
The developer configures one or more thread pools responsible for Asynchronous
Operations with a particular type of dependency\footnote{Thread pools may be
associated with more than one dependency type.}.  The Asynchronous Operation is
then matched by its required dependency types to a thread pool responsible for
one or more of its dependency types\footnote{The \texttt{operationId} may also
be used for very fine grained mapping.}.  The \texttt{execute(\ldots)} method of
the Asynchronous Operation (Fig \ref{fig:InjectionInterfaces}) is then invoked
by a thread from the matching thread pool to execute the implementation of the
Asynchronous Operation.  This provides the necessary isolation of differing
Asynchronous Operations for better tuning of the Web Server.

For Asynchronous Operations not having dependencies (or dependencies of any
performance significance), a default thread pool is configured by the developer
for their execution.  This ensures all Asynchronous Operations are mapped to a
thread pool.  It also means that thread pools need only be configured for
dependencies requiring isolation.

The \texttt{cancel(Exception)} method provides the means for the Asynchronous
Operation Processor to cancel the Asynchronous Operation.  The Asynchronous
Operation Processor will cancel new Asynchronous Operations for a particular
thread pool when queuing the Asynchronous Operation for a thread will result in
exceeding a particular threshold\footnote{A sufficient threshold would be
ensuring the wait time, determined by a running average of execution time
multiplied against the number of Asynchronous Operations currently in the queue,
is below a certain time.}.  Each thread pool may have its own thresholds
particular to its responsible dependency types.  As Asynchronous Operations are
mapped to particular Thread Pools, this ensures only the appropriate operations
are cancelled.

Once cancelled, the Asynchronous Operation Processor may discard the
Asynchronous Operation.  The implementation of the \texttt{cancel(Exception)}
method is specific to each Asynchronous Operation.  This is to allow appropriate
application behaviour to occur on cancelling Asynchronous Operations.


\subsection{Example}

Table \ref{tab:example_request_thread_pools} provides an example set of thread
pools for the motivating example.  A thread pool is configured for Asynchronous
Operations with a \texttt{Database connection}.  As the \texttt{Database
connection} is only available via extrinsic Dependency Injection, all
Asynchronous Operations requiring I/O with the database must depend on the
\texttt{Database connection} and are subsequently mapped to be executed by a
thread from the \texttt{Database} thread pool.  The remaining operations have
their own thread pools allowing them to continue to be executed even if the
database I/O is causing thread starvation within its own thread pool.

\begin{table}[t]
\tbl{Example of assigning Asynchronous Operations to thread pools by dependency type}{%
\begin{tabular}{|l||l||l|}
\hline
\bfseries Thread Pool & \bfseries Dependency Type & \bfseries Asynchronous Operation \\
\hline
Network & Selector & Read data from socket \\
\hline
Database & Database connection & Retrieve data from database \\
\hline
Default & - & Parse HTTP request, \\
& & Dispatch HTTP request, \\
& & Validate client data, \\ 
& & Retrieve data from cache, \\
& & Render HTTP response, \\
& & Write HTTP response \\
\hline
\end{tabular}}
\label{tab:example_request_thread_pools}
\end{table}

The result for the motivating example is that the operations for servicing the
cached request can now be prioritized.  This can occur even if the database
driver has become non-responsive blocking all threads attempting to use a
database connection.  Furthermore, as the application becomes more complex with
an increasing number of downstream systems (e.g. Reverse 10K problem
\cite{reverse-ten-k-problem}), each downstream system's performance impacts may
be isolated by assigning it its own thread pool.  This has requests requiring a
slow downstream system to be deprioritized, allowing other requests to be
serviced.


\subsection{Consequences}

%% TODO: negative consequences? (more intricate understanding of application as developer involved in configuring threading.  It is not defaulted to a single thread pool which is adequate for most low load servers and relieves the burdon on the developer to understand thread tuning.)

Thread Injection provides the implementing algorithm for the Proactor pattern's
Asynchronous Operation Processor participant by using multiple Thread Pools to
prioritise and cancel Asynchronous Operations.

Improved tuning of the Web Server is possible by the isolation provided by using
multiple thread pools.  Tuning the thread pools (such as restricting the number
of threads or changing the pool's thread nice values) allows prioritizing
threads and subsequently prioritizing groups of related Asynchronous Operations.

As each thread pool is executing Asynchronous Operations for a particular
dependency (or set of dependencies), this allows for adaptive resource
management and admission control regarding the dependency \cite{seda}.  This
enables both the number of threads and dependencies to be dynamically altered to
improve throughput.  However, when maximum throughput is reached additional
Asynchronous Operations above this threshold can be cancelled.

To improve performance of runtime decisions, the mapping of Asynchronous
Operation to thread pool may be cached.  As the dependencies for each
Asynchronous Operation is static, at application start up time the Asynchronous
Operation Processor may preprocess and cache the mapping of Asynchronous
Operations to thread pools to reduce runtime decision overheads.  This
pre-mapping of Asynchronous Operations may also provide warnings where
dependencies of an Asynchronous Operation make it possible to map the
Asynchronous Operation to multiple thread pools.  Different conflict mapping
resolutions may be employed, however ordering the thread pools and assigning
Asynchronous Operations based on first match is a sufficiently adequate
algorithm.

From the perspective of the Asynchronous Operation, the developer is able to
configure which thread pool will execute each Asynchronous Operation.  This
effectively injects the thread for execution in a similar way to injecting a
dependency for use.


\subsection{Related Patterns}

The Thread Injection pattern utilises information provided by the Dependency
Injection pattern.  The Thread Injection can be used in isolation of the
Dependency Injection pattern by mapping each Asynchronous Operation directly to
a thread pool and using intrinsic dependencies.  This however involves increased
configuration for the developer and may become invalid should Asynchronous
Operations change their dependencies as the application evolves.  It is
therefore recommended to use Thread Injection in combination with Dependency
Injection to map Asynchronous Operations by dependencies to thread pools to
reduce configuration and mapping errors.

Utilising Thread Injection in combination with Continuation Injection allows
reducing thread-context switching of the application.  As Continuation Injection
allows Asynchronous Operations to be synchronously invoked, the thread may be
borrowed to avoid the need for a thread-context switch.

The Thread Pool pattern \cite{thread-per-request} is used to reduce overheads of
thread management for improved efficiency in execution of the thread pool's
responsible Asynchronous Operations.  To achieve further efficiencies, the
implementation of the thread pool may be specific to the dependency.  For
example, the thread pool may contain multiple threads for concurrent execution
of Asynchronous Operations, a single thread for serial execution of Asynchronous
Operations or even no threads and execute Asynchronous Operations by borrowing
the thread to reduce thread-context switching.


\subsection{Known Usage}

The Thread Injection pattern can be considered a style of cohort scheduling
\cite{cohort} that groups operations with similar dependencies and infers from
that similar functionality.  However, Thread Injection works at the application
scheduling level and allows use of any Operating System thread scheduling
algorithms.

The Staged Event-Driven Architecture (SEDA) \cite{seda} provides an
implementation of the Thread Injection pattern without the use of the Dependency
Injection pattern.  SEDA directly maps operations to a stage and subsequently a
thread pool.  The SEDA pipeline however has increased thread-context switching
as the stage boundaries are hard disallowing threads to be borrowed.

Dependency capsules \cite{dependency-capsules} follows the idea of isolating
operations requiring dependencies to specific thread pools.  It however requires
a thread-context switch back to a main thread for executing operations without
dependencies.

OfficeFloor \cite{officefloor} implements the Thread Injection pattern and will
be discussed later in this paper as it also implements the Continuation
Injection pattern to provide an Inversion of Control implementation.



\section{Inversion of Control}

%% TODO: problem/solution matching?

\subsection{Problem}

\textbf{How to build a framework that loosely couples components?}


\subsection{Context}

%% TODO: Maybe the exact design situation where this pattern is needed should be mentioned in the end of the context.

%% TODO: avoid abstract methods that bind the signature of the implementation and thread of control (reference Inversion of Control layer)

Object-orientation realises an application as a graph of objects modelling the
data.  Objects model data and are weaker at modelling the functionality of an
application \cite{oo-behaviour}.  Using object design techniques can also make
designing applications more difficult to less experienced developers
\cite{oo-design}.

Bridging object-orientated programming with functional programming
\cite{bridging-function-oo} compensates for the weakness of
objection-orientation in modelling functionality.  Example languages bridging
object-orientation and functional programming are Scala \cite{scala} and F\#
\cite{f-sharp}.

While this evolution is making programming languages richer in expression and
features, it continues to interface components by method/function signatures.
``The set of all [method] signatures defined by an object \ldots characterizes
the complete set of requests that can be sent to the object'' \cite[p. 13]{gof}.
 However unlike the Actor Model \cite{actors}, which decouples the sender from
the message (request), the method/function signature tightly couples the
components.

As the method/function signature is controlled by the developer, good design is
necessary to avoid tight coupling.  ``Tight coupling leads to monolithic
systems, where you can't change or remove a class[/component] without
understanding and changing other classes[/components]'' \cite[p. 24-25]{gof}.
Having to change components due to poor developer design negates the ability to
assemble an application from discrete, reusable components.

To avoid tight coupling of discrete components, the design over a component's
interface should be removed from the component's control. ``Object[/component]
composition requires that the objects[/components] being composed have
well-defined interfaces \ldots which in turn requires carefully designed
interfaces'' \cite[p. 19]{gof}.  Rather than leaving the component interface to
good design, the interface for all components should be standardised to avoid
the possibility of bad design and subsequently tight coupling.

Having a standard interface to all components that provides loose coupling will
enable assembly of applications from discrete components.  This will enable
reuse of components that leads to inversion of control \cite[p. 27]{gof}.


\subsection{Forces}

The components need to be composed within an architecture which is dictated by
the framework.  The framework ``will define the overall structure, its
partitioning into \ldots [components], the key responsibilities thereof, how the
\ldots [components] collaborate, and the thread of control'' \cite[p.26]{gof}. 
Using the components within a framework, therefore, identifies the following
requirements of standardising the component interface:
\begin{itemize}
  \item components must have a key responsibility,
  \item components must be able to collaborate with other components, and
  \item components require a thread of control.
\end{itemize}

The collaboration of components can further be defined as the following
requirements:
\begin{itemize}
  \item invoking other components,
  \item sharing state with the other components, and
  \item handling exceptions from components.
\end{itemize}

The method/function signature is a component interface that fulfills each
requirement.  The method/function signature:
\begin{itemize}
  \item has a key responsibility identified by its name,
  \item may invoke other methods/functions,
  \item shares state with other methods/functions by arguments and return values,
  \item provides declaration of exceptions, and
  \item can be executed by the thread of control\footnote{For a standard component interface, thread local variables are to be avoided to not incur affinity to a thread.}.
\end{itemize}

The method/function interface is, however, subject to tight coupling.  The tight
coupling occurs from the invoking component having to:
\begin{itemize}
  \item be aware of the method/function name,
  \item provide the necessary arguments,
  \item possibly use the return value,
  \item handle potential exceptions, and
  \item use its thread of control to execute the method/function.
\end{itemize}

Providing an interface that decouples these aspects of the method/function
signature from the invoking component will provide the required discrete
component interface.


\subsection{Solution}

%% TODO: solution should have an illustration

\textbf{Use Dependency Injection, Thread Injection and Continuation Injection to manage state, execution and collaboration of the components to subsequently provide a standard, loosely coupled interface to the components.}

Continuation Injection enables invoking the other components (Asynchronous
Operations) and uses indirection to avoid coupling the name of the invoked
component.  The mapping of \texttt{continuationId} to \texttt{operationId}
avoids coupling the names.

Extrinsic Dependency Injection \cite{ioc} provides sharing of state for both
arguments and return values of components.  Using the Dependency Injection
pattern for the creation of both Asynchronous Operations and their dependencies
results in both Asynchronous Operations and dependencies residing in the same
context (e.g. context for servicing a request).  Having the dependencies in the
same context as the Asynchronous Operations allows the dependencies to be
re-used across the Asynchronous Operations.  Hence, managing the state of the
application within the dependencies shares state across Asynchronous Operations
by the re-use of dependencies.  Therefore, dependencies are used as arguments
for the components\footnote{To further decouple the state sharing for the
Asynchronous Operation, the \texttt{AsynchronousOperationContext} may contain
developer configuration to map the \texttt{dependencyId} to the context specific
identifier for the dependency.}.  Furthermore as collaboration is asynchronous
via Continuation Injection, there is no need for return values.  However, return
values are achieved by changing state of a dependency and having the invoking
component use the dependency on completion of the continuation \texttt{Future}. 
This further decouples the invoking component, as it may obtain multiple return
values via multiple dependencies.

Continuation Injection handles exceptions by invoking other components
(Asynchronous Operations). The exception is mapped to a \texttt{continuationId}
and subsequently a handling Asynchronous Operation.  The exception may be used
as the continuation parameter.  Also, as other Asynchronous Operations are
handling the exceptions, the invoking component is further decoupled by not
having to handle exceptions.

Thread Injection provides selection of the thread of control for the component
(Asynchronous Operation).  The component may borrow the thread or it may use a
thread from a particular thread pool.

Using the Dependency Injection, Thread Injection and Continuation Injection
patterns together provides the necessary decoupling of components required as:
\begin{itemize}
  \item decoupling method/function name is by Continuation Injection,
  \item providing the necessary arguments is by Dependency Injection,
  \item possibly using the return value is by Dependency Injection and Continuation Injection,
  \item handling potential exceptions is by Continuation Injection, and
  \item the thread of control is decoupled by Thread Injection.
\end{itemize}

Therefore, the \texttt{AsynchronousOperation} interface
(Fig\ref{fig:InjectionInterfaces}) is the decoupled standard interface for a
component.  As the component has a decoupled standardised interface, the
application may be assembled from reusable components built to this interface. 
Furthermore, as the framework is managing state, execution and collaboration of
the components, the framework is providing inversion of control\footnote{The
Inversion of Control pattern can be described as: Inversion of Control =
Dependency Injection + Thread Injection + Continuation Injection (or its shorter
form: IoC = DI + TI + CI).} over the components.


\subsection{Example}

The example demonstrates the Inversion of Control pattern resolving the
motivating example.  The \texttt{Retrieve data from cache} operation (Table
\ref{tab:example_request_operations}) from the motivating example will be used.  

As the Thread Injection and Continuation Injection patterns define the
algorithms for all Proactor pattern participants (except the application
behaviour of the Asynchronous Operation), only the Asynchronous Operation
developer implementation will be provided.  Figure
\ref{fig:Example_Method_Operation} shows the example developer implementation
code.  A generic Asynchronous Operation adapter implementation is used to
reflectively invoke the \texttt{retrieveData(\ldots)} method. It will:
\begin{enumerate}
  \item Obtain an instance of the \texttt{CacheOperation} via its context's \texttt{getDependency(\ldots)} method.
  \item Obtain both the \texttt{key}\footnote{\texttt{key} is a continuation argument from the previous Asynchronous Operation.} and \texttt{cache} again via the \texttt{getDependency(\ldots)} method.
  \item Instantiate a proxy implementation of the \texttt{CacheContinuation} interface that implements the \texttt{cacheMiss(\ldots)} method by invoking the \texttt{doContinuation(\ldots)} method. 
  \item Reflectively invokes the \texttt{retrieveData(\ldots)} with the above arguments.
\end{enumerate}

Continuations from the \texttt{retrieveData(\ldots)} method are:
\begin{itemize}
  \item \texttt{cacheMiss(\ldots)} which will be mapped to the \texttt{Retrieve data from database} Asynchronous Operation.
  \item Implicit continuation (continuation invoked if no other continuation is invoked) which is mapped to the \texttt{Render HTTP response}\footnote{The return value from the method is used as the continuation argument.}.
  \item \texttt{IOException} that is mapped to an Asynchronous Operation providing an error message page.  It may also be mapped to the \texttt{Retrieve data from database} Asynchronous Operation to attempt to continue servicing the request.
  \item A default continuation is used should the Asynchronous Operation be cancelled.
\end{itemize}

The thread of control is configured based on the \texttt{Cache} implementation.
 Should the \texttt{cache} be in local memory it will not have a specific thread
pool configured.  The \texttt{retrieveData(\ldots)} method will subsequently be
executed by borrowing the thread.  However, should the \texttt{cache} be
remotely located requiring network access, a thread pool would be assigned to
it.  The \texttt{retrieveData(\ldots)} method will be executed by a thread from
this thread pool.

This example implementation of an Asynchronous Operation demonstrates resolution
of the issues within the motivating example.  The operation is a discrete
method.  It specifies the next operation (e.g. skipping the \texttt{Retrieve
data from database} operation and going straight to the \texttt{Render HTTP
response} operation) by invoking the appropriate continuation.  Thread-context
switching is reduced by borrowing the thread when no dependencies require
isolation.  Note the \texttt{Retrieve data from database} (table
\ref{tab:example_request_operations}) operation would be isolated to a thread
pool by the \texttt{Database} dependency.


\subsection{Consequences}

Object-orientation provides the implicit dependencies to the method.  Explicit
dependencies are injected as arguments to the method, while object-orientation
implicitly provides the instance/class references to the method.  Functional
programming can therefore be considered programming without implicit
dependencies.

Furthermore, Object-orientation can focus on modelling the data and have its
methods constrain the changes in state.  This is similar to databases storing
data and constraining the changes in the data.  As application behaviour is
implemented in Asynchronous Operations, the application need not be constructed
only as a graph of objects.

As the Thread Injection and Continuation Injection patterns provide the
implementing algorithms of the Proactor pattern participants, the Inversion of
Control pattern has provided limited improvement regarding the Proactor pattern
being ``hard to debug'' \cite[p. 7]{proactor}.  While the patterns have not
directly addressed the Proactor issue of being hard to debug, they do follow the
trend of popular modern Web Servers by encapsulating developer code within
Asynchronous Operations.  However, the introduction of dependencies maintaining
state between Asynchronous Operations executed by different threads leaves the
possibility for dependencies to be non-thread safe causing difficult to identify
thread synchronization bugs.  However, as application logic is contained in
Asynchronous Operations, there is better re-use of dependencies which improves
their quality and subsequently reduces the risk of these synchronization bugs.

As demonstrated in the example, control may be inverted for all interfacing
aspects of the method.  This has potentially made the Asynchronous Operation the
software equivalent of a \textit{brick}\footnote{Objects/methods/functions do
not define the dimensions of execution and asynchronous collaboration providing
only part of the ingredients to a building block (brick).}.  The Asynchronous
Operations have standard interface for all their dimensions (state, execution
and collaboration) providing a standard mechanism for developers to weave the
\textit{bricks} (Asynchronous Operations) together to create the equivalent of
\textit{walls}, \textit{buildings} and further complex
structures\footnote{Discussion on the patterns that weave Asynchronous
Operations into more complex, re-usable components will be left to further
work.}.


\subsection{Related Patterns}

The Inversion of Control pattern is realised by using the Dependency Injection,
Thread Injection and Continuation Injection patterns as implementing algorithms
to the Proactor pattern.

The Adapter pattern \cite{gof} adapts the interfaces of dependencies.
Standardising the dependency interfaces across all Asynchronous Operations may
not be feasible.  The Adaptor pattern allows re-use of the same dependency by
adapting its interface for the particular Asynchronous Operation.  Dependency
Injection enables the adapted dependency by providing a new dependency with the
required interface that depends on (wraps) the re-used dependency.


\subsection{Known Usage}

OfficeFloor \cite{officefloor} identifies Asynchronous Operations as
Jobs\footnote{Task is the adapted interface within a Job for application
developers to implement.}, which originates from OfficeFloor's modelling of
application architecture after the way work is processed within an
office\footnote{OfficeFloor derived its name from being the place containing the
co-ordination of executing Jobs.}.  Jobs are:
\begin{itemize}
  \item undertaken by Teams (particular thread pools via Thread Injection),
  \item may access Managed Objects (explicit dependencies via extrinsic Dependency Injection), and
  \item invoke flows (continuations providing interaction of Jobs via Continuation Injection).
\end{itemize}

OfficeFloor is the only known framework implementing the Inversion of Control
pattern.  Through developing OfficeFloor, the author identified the Thread
Injection and Continuation Injection patterns along with the subsequent
Inversion of Control pattern.



\section{Web Server implementation}

This section discusses using the Thread Injection, Continuation Injection and
Inversion of Control patterns in implementing a Web Server.


\subsection{Servicing requests and managing URLs}

For Web Servers, URL continuations \cite{url-continuation} are used to invoke
the first Asynchronous Operation for servicing the request\footnote{A similar
approach may be taken by other User Interface types (such as a rich GUI) with
the user event being mapped to the first Asynchronous Operation.}.  A URL may be
associated to an Asynchronous Operation by developer configuration.  This
Asynchronous Operation then becomes the first operation to be invoked for
servicing a HTTP request for that URL.  This configuration may be included in
the Continuation Injection configuration (Operation Orchestration) as an
attribute of the Asynchronous Operation node.

Providing identifiers to Asynchronous Operations is also a convenient means to
ease maintenance of web pages.  Rather than embedding the URL in the web page,
the Asynchronous Operation identifier is used.  When the page is rendered for
the client the identifiers are replaced with the actual URLs.  This allows the
URLs to be changed without needing to change web pages.  It also enables using
configuration to change particular URLs to use secure channels (e.g. HTTPS)
without requiring changes to web pages.

Substituting the Asynchronous Operation URLs at page rendering time also enables
distributed load balancing.  A server may direct clients to a different server
by substituting URLs for Asynchronous Operations on another server. The
appropriate clients will then continue with the other server and subsequently
allow balancing the load across the distributed servers.


\subsection{Continuation for cancelling an Asynchronous Operation}

Cancelling the Asynchronous Operation is handled by the Asynchronous Operation
invoking a continuation.  On the \texttt{cancel(Exception)} method (Fig
\ref{fig:InjectionInterfaces}) being executed, the Asynchronous Operation will
clean up any resources and then invoke a continuation for an Asynchronous
Operation executed in another thread pool.  For a Web Server this Asynchronous
Operation could for example send a web page indicating the server is temporarily
busy or it may send a redirect to a less busy Web Server.

Each Asynchronous Operation may individually specify the continuation or a
default continuation may be configured across a set of Asynchronous Operations.
The default continuation will be mapped to an Asynchronous Operation that
interrogates the \texttt{cause} and undertakes appropriate further Asynchronous
Operations specific to the required application behaviour.  This is very similar
to \texttt{catch} blocks for handling an \texttt{Exception}.


\subsection{Implicit Thread}

When using the Thread Injection and Continuation Injection patterns together,
implicit threads may be used to reduce thread-context switching.

The continuation may borrow the thread of the invoking Asynchronous Operation if
it results in being executed by the same thread pool.  The Proactor Pattern
stipulates that Asynchronous Operations ``must be performed without borrowing
the application's thread of control'' \cite[p. 8]{proactor} and this is the
focus of the Reactor Pattern \cite{reactor}.  Rather than dispatching back to
the same thread pool, the continued Asynchronous Operation may borrow the thread
to avoid the overhead of a thread-context switch.

This same idea is also extended to the default thread pool for the Thread
Injection pattern.  Asynchronous Operations mapped to the default thread pool do
not have dependencies requiring isolation and are deemed cheaper to be executed
by borrowing the thread rather than incurring the cost of a thread-context
switch.

The borrowed thread is the implicit thread.  Like an implicit continuation that
executes the next operation \cite{continuations}, the implicit thread executes
the next operation unless an explicit thread is required.  This can result in
the Web Server servicing the entire request without a thread-context switch
should all the Asynchronous Operations involved not require an explicit thread
(e.g. web page content obtained from cache).

An implicit thread has similarities to a monadic thread \cite{monadic-thread}.
The Asynchronous Operations can be considered nodes in the lazy trace of the
monadic thread.  The advantage of an implicit thread\footnote{Beyond Operation
Orchestration being easier for the developer to understand than monad
programming.} is that Thread Injection allows prioritizing the execution of
blocking I/O nodes (Asynchronous Operations) by using explicit threads.  Monadic
threads can not prioritize blocking I/O nodes as they know little about them and
subsequently execute them within a single thread pool.


\subsection{Clustering/Distributing Asynchronous Operations}

As continuations asynchronously invoke Asynchronous Operations, the Asynchronous
Operation need not reside on the same server.  The injected continuation can be
configured to invoke an Asynchronous Operation on another server.  The
continuation may synchronously send the continuation arguments (e.g. HTTP
request) or the arguments may be asynchronously communicated (e.g.
queue).

Distributing Asynchronous Operations heterogeneously can also be undertaken by
configuring the responsible thread pool to reside on another server.  On
invoking the continuation, the continuation arguments are communicated to the
server\footnote{May be a cluster of servers behind a load balancer.} containing
the responsible thread pool.  The dependency injection context state, to be
reinstated on the other server, may also be communicated with the continuation
arguments depending on thread pool mapping configuration.  Mapping the
responsible thread pool to another server is useful for example to co-locate
certain Asynchronous Operations geographically or have certain Asynchronous
Operations run on particular hardware.

The Internet is an example implementation that utilizes both Asynchronous
Operation heterogeneous clustering and the invoking of continuations to other
servers.  The URL continuation (e.g. by a user clicking on a link) is sent to
the server containing the thread pool (i.e. Web Server).  The Web Server
continues the processing by sending the HTTP response to the client web browser
(server to the user who may trigger further continuations).

The evolution to Cloud Computing can also be described as a form of Asynchronous
Operation heterogeneous clustering.  The number of servers containing a
particular thread pool are dynamically managed based on the number of
Asynchronous Operations to execute.  This dynamic management enables only the
required number of servers to be allocated for each thread pool.



\section{Future work}

The patterns presented in this paper describe how OfficeFloor \cite{officefloor}
provides an Inversion of Control implementation.  Future work will describe the
patterns used by OfficeFloor to weave Asynchronous Operations (Jobs) together
into re-usable, composite components (Sections).  Patterns will also be
presented on how OfficeFloor weaves these composite components into greater
composite components (Offices and OfficeFloors) to manage the complexity of an
application.



\section*{Acknowledgment} I thank my wife Melanie for her patience and support
of me developing OfficeFloor on top of my day job.  If she was anyone else
OfficeFloor would not have been built and this work would not have resulted from
OfficeFloor.  I also thank my good friend Matthew Brown for being a sounding
board to many of my ideas.

I am also grateful for the wise shepherding by Veli-Pekka Eloranta to ensure
this paper succinctly presents the Thread Injection, Continuation Injection and
Inversion of Control patterns.


\bibliographystyle{style/acmlarge}
\bibliography{tici}

\end{document}
