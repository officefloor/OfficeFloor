%% OfficeFloor - http://www.officefloor.net
%% Copyright (C) 2012 Daniel Sagenschneider
%%
%% This program is free software: you can redistribute it and/or modify
%% it under the terms of the GNU General Public License as published by
%% the Free Software Foundation, either version 3 of the License, or
%% (at your option) any later version.
%%
%% This program is distributed in the hope that it will be useful,
%% but WITHOUT ANY WARRANTY; without even the implied warranty of
%% MERCHANTABILITY or FITNESS FOR A PARTICULAR PURPOSE.  See the
%% GNU General Public License for more details.
%%
%% You should have received a copy of the GNU General Public License
%% along with this program.  If not, see <http://www.gnu.org/licenses/>.
%%
%% While this document is not a program, it conveys the underlying design of 
%% OfficeFloor and as such any program derived from these ideas is considered 
%% conveying OfficeFloor and is subject to the licensing of OfficeFloor. 


%% bare_conf.tex
%% V1.3
%% 2007/01/11
%% by Michael Shell
%% See:
%% http://www.michaelshell.org/
%% for current contact information.
%%
%% This is a skeleton file demonstrating the use of IEEEtran.cls
%% (requires IEEEtran.cls version 1.7 or later) with an IEEE conference paper.
%%
%% Support sites:
%% http://www.michaelshell.org/tex/ieeetran/
%% http://www.ctan.org/tex-archive/macros/latex/contrib/IEEEtran/
%% and
%% http://www.ieee.org/

%%*************************************************************************
%% Legal Notice:
%% This code is offered as-is without any warranty either expressed or
%% implied; without even the implied warranty of MERCHANTABILITY or
%% FITNESS FOR A PARTICULAR PURPOSE! 
%% User assumes all risk.
%% In no event shall IEEE or any contributor to this code be liable for
%% any damages or losses, including, but not limited to, incidental,
%% consequential, or any other damages, resulting from the use or misuse
%% of any information contained here.
%%
%% All comments are the opinions of their respective authors and are not
%% necessarily endorsed by the IEEE.
%%
%% This work is distributed under the LaTeX Project Public License (LPPL)
%% ( http://www.latex-project.org/ ) version 1.3, and may be freely used,
%% distributed and modified. A copy of the LPPL, version 1.3, is included
%% in the base LaTeX documentation of all distributions of LaTeX released
%% 2003/12/01 or later.
%% Retain all contribution notices and credits.
%% ** Modified files should be clearly indicated as such, including  **
%% ** renaming them and changing author support contact information. **
%%
%% File list of work: IEEEtran.cls, IEEEtran_HOWTO.pdf, bare_adv.tex,
%%                    bare_conf.tex, bare_jrnl.tex, bare_jrnl_compsoc.tex
%%*************************************************************************

% *** Authors should verify (and, if needed, correct) their LaTeX system  ***
% *** with the testflow diagnostic prior to trusting their LaTeX platform ***
% *** with production work. IEEE's font choices can trigger bugs that do  ***
% *** not appear when using other class files.                            ***
% The testflow support page is at:
% http://www.michaelshell.org/tex/testflow/



% Note that the a4paper option is mainly intended so that authors in
% countries using A4 can easily print to A4 and see how their papers will
% look in print - the typesetting of the document will not typically be
% affected with changes in paper size (but the bottom and side margins will).
% Use the testflow package mentioned above to verify correct handling of
% both paper sizes by the user's LaTeX system.
%
% Also note that the "draftcls" or "draftclsnofoot", not "draft", option
% should be used if it is desired that the figures are to be displayed in
% draft mode.
%
\documentclass[conference]{ieee/IEEEtran}
% Add the compsoc option for Computer Society conferences.
%
% If IEEEtran.cls has not been installed into the LaTeX system files,
% manually specify the path to it like:
% \documentclass[conference]{../sty/IEEEtran}

 



% Some very useful LaTeX packages include:
% (uncomment the ones you want to load)


% *** MISC UTILITY PACKAGES ***
%
%\usepackage{ifpdf}
% Heiko Oberdiek's ifpdf.sty is very useful if you need conditional
% compilation based on whether the output is pdf or dvi.
% usage:
% \ifpdf
%   % pdf code
% \else
%   % dvi code
% \fi
% The latest version of ifpdf.sty can be obtained from:
% http://www.ctan.org/tex-archive/macros/latex/contrib/oberdiek/
% Also, note that IEEEtran.cls V1.7 and later provides a builtin
% \ifCLASSINFOpdf conditional that works the same way.
% When switching from latex to pdflatex and vice-versa, the compiler may
% have to be run twice to clear warning/error messages.






% *** CITATION PACKAGES ***
%
\usepackage{cite}
% cite.sty was written by Donald Arseneau
% V1.6 and later of IEEEtran pre-defines the format of the cite.sty package
% \cite{} output to follow that of IEEE. Loading the cite package will
% result in citation numbers being automatically sorted and properly
% "compressed/ranged". e.g., [1], [9], [2], [7], [5], [6] without using
% cite.sty will become [1], [2], [5]--[7], [9] using cite.sty. cite.sty's
% \cite will automatically add leading space, if needed. Use cite.sty's
% noadjust option (cite.sty V3.8 and later) if you want to turn this off.
% cite.sty is already installed on most LaTeX systems. Be sure and use
% version 4.0 (2003-05-27) and later if using hyperref.sty. cite.sty does
% not currently provide for hyperlinked citations.
% The latest version can be obtained at:
% http://www.ctan.org/tex-archive/macros/latex/contrib/cite/
% The documentation is contained in the cite.sty file itself.






% *** GRAPHICS RELATED PACKAGES ***
%
\ifCLASSINFOpdf
   \usepackage[pdftex]{graphicx}
  % declare the path(s) where your graphic files are
   \graphicspath{{./pdf/}}
  % and their extensions so you won't have to specify these with
  % every instance of \includegraphics
   \DeclareGraphicsExtensions{.pdf}
\else
  % or other class option (dvipsone, dvipdf, if not using dvips). graphicx
  % will default to the driver specified in the system graphics.cfg if no
  % driver is specified.
  % \usepackage[dvips]{graphicx}
  % declare the path(s) where your graphic files are
  % \graphicspath{{../eps/}}
  % and their extensions so you won't have to specify these with
  % every instance of \includegraphics
  % \DeclareGraphicsExtensions{.eps}
\fi
% graphicx was written by David Carlisle and Sebastian Rahtz. It is
% required if you want graphics, photos, etc. graphicx.sty is already
% installed on most LaTeX systems. The latest version and documentation can
% be obtained at: 
% http://www.ctan.org/tex-archive/macros/latex/required/graphics/
% Another good source of documentation is "Using Imported Graphics in
% LaTeX2e" by Keith Reckdahl which can be found as epslatex.ps or
% epslatex.pdf at: http://www.ctan.org/tex-archive/info/
%
% latex, and pdflatex in dvi mode, support graphics in encapsulated
% postscript (.eps) format. pdflatex in pdf mode supports graphics
% in .pdf, .jpeg, .png and .mps (metapost) formats. Users should ensure
% that all non-photo figures use a vector format (.eps, .pdf, .mps) and
% not a bitmapped formats (.jpeg, .png). IEEE frowns on bitmapped formats
% which can result in "jaggedy"/blurry rendering of lines and letters as
% well as large increases in file sizes.
%
% You can find documentation about the pdfTeX application at:
% http://www.tug.org/applications/pdftex





% *** MATH PACKAGES ***
%
%\usepackage[cmex10]{amsmath}
% A popular package from the American Mathematical Society that provides
% many useful and powerful commands for dealing with mathematics. If using
% it, be sure to load this package with the cmex10 option to ensure that
% only type 1 fonts will utilized at all point sizes. Without this option,
% it is possible that some math symbols, particularly those within
% footnotes, will be rendered in bitmap form which will result in a
% document that can not be IEEE Xplore compliant!
%
% Also, note that the amsmath package sets \interdisplaylinepenalty to 10000
% thus preventing page breaks from occurring within multiline equations. Use:
%\interdisplaylinepenalty=2500
% after loading amsmath to restore such page breaks as IEEEtran.cls normally
% does. amsmath.sty is already installed on most LaTeX systems. The latest
% version and documentation can be obtained at:
% http://www.ctan.org/tex-archive/macros/latex/required/amslatex/math/





% *** SPECIALIZED LIST PACKAGES ***
%
%\usepackage{algorithmic}
% algorithmic.sty was written by Peter Williams and Rogerio Brito.
% This package provides an algorithmic environment fo describing algorithms.
% You can use the algorithmic environment in-text or within a figure
% environment to provide for a floating algorithm. Do NOT use the algorithm
% floating environment provided by algorithm.sty (by the same authors) or
% algorithm2e.sty (by Christophe Fiorio) as IEEE does not use dedicated
% algorithm float types and packages that provide these will not provide
% correct IEEE style captions. The latest version and documentation of
% algorithmic.sty can be obtained at:
% http://www.ctan.org/tex-archive/macros/latex/contrib/algorithms/
% There is also a support site at:
% http://algorithms.berlios.de/index.html
% Also of interest may be the (relatively newer and more customizable)
% algorithmicx.sty package by Szasz Janos:
% http://www.ctan.org/tex-archive/macros/latex/contrib/algorithmicx/




% *** ALIGNMENT PACKAGES ***
%
\usepackage{array}
% Frank Mittelbach's and David Carlisle's array.sty patches and improves
% the standard LaTeX2e array and tabular environments to provide better
% appearance and additional user controls. As the default LaTeX2e table
% generation code is lacking to the point of almost being broken with
% respect to the quality of the end results, all users are strongly
% advised to use an enhanced (at the very least that provided by array.sty)
% set of table tools. array.sty is already installed on most systems. The
% latest version and documentation can be obtained at:
% http://www.ctan.org/tex-archive/macros/latex/required/tools/


%\usepackage{mdwmath}
%\usepackage{mdwtab}
% Also highly recommended is Mark Wooding's extremely powerful MDW tools,
% especially mdwmath.sty and mdwtab.sty which are used to format equations
% and tables, respectively. The MDWtools set is already installed on most
% LaTeX systems. The lastest version and documentation is available at:
% http://www.ctan.org/tex-archive/macros/latex/contrib/mdwtools/


% IEEEtran contains the IEEEeqnarray family of commands that can be used to
% generate multiline equations as well as matrices, tables, etc., of high
% quality.


%\usepackage{eqparbox}
% Also of notable interest is Scott Pakin's eqparbox package for creating
% (automatically sized) equal width boxes - aka "natural width parboxes".
% Available at:
% http://www.ctan.org/tex-archive/macros/latex/contrib/eqparbox/





% *** SUBFIGURE PACKAGES ***
%\usepackage[tight,footnotesize]{subfigure}
% subfigure.sty was written by Steven Douglas Cochran. This package makes it
% easy to put subfigures in your figures. e.g., "Figure 1a and 1b". For IEEE
% work, it is a good idea to load it with the tight package option to reduce
% the amount of white space around the subfigures. subfigure.sty is already
% installed on most LaTeX systems. The latest version and documentation can
% be obtained at:
% http://www.ctan.org/tex-archive/obsolete/macros/latex/contrib/subfigure/
% subfigure.sty has been superceeded by subfig.sty.



%\usepackage[caption=false]{caption}
%\usepackage[font=footnotesize]{subfig}
% subfig.sty, also written by Steven Douglas Cochran, is the modern
% replacement for subfigure.sty. However, subfig.sty requires and
% automatically loads Axel Sommerfeldt's caption.sty which will override
% IEEEtran.cls handling of captions and this will result in nonIEEE style
% figure/table captions. To prevent this problem, be sure and preload
% caption.sty with its "caption=false" package option. This is will preserve
% IEEEtran.cls handing of captions. Version 1.3 (2005/06/28) and later 
% (recommended due to many improvements over 1.2) of subfig.sty supports
% the caption=false option directly:
%\usepackage[caption=false,font=footnotesize]{subfig}
%
% The latest version and documentation can be obtained at:
% http://www.ctan.org/tex-archive/macros/latex/contrib/subfig/
% The latest version and documentation of caption.sty can be obtained at:
% http://www.ctan.org/tex-archive/macros/latex/contrib/caption/




% *** FLOAT PACKAGES ***
%
%\usepackage{fixltx2e}
% fixltx2e, the successor to the earlier fix2col.sty, was written by
% Frank Mittelbach and David Carlisle. This package corrects a few problems
% in the LaTeX2e kernel, the most notable of which is that in current
% LaTeX2e releases, the ordering of single and double column floats is not
% guaranteed to be preserved. Thus, an unpatched LaTeX2e can allow a
% single column figure to be placed prior to an earlier double column
% figure. The latest version and documentation can be found at:
% http://www.ctan.org/tex-archive/macros/latex/base/



%\usepackage{stfloats}
% stfloats.sty was written by Sigitas Tolusis. This package gives LaTeX2e
% the ability to do double column floats at the bottom of the page as well
% as the top. (e.g., "\begin{figure*}[!b]" is not normally possible in
% LaTeX2e). It also provides a command:
%\fnbelowfloat
% to enable the placement of footnotes below bottom floats (the standard
% LaTeX2e kernel puts them above bottom floats). This is an invasive package
% which rewrites many portions of the LaTeX2e float routines. It may not work
% with other packages that modify the LaTeX2e float routines. The latest
% version and documentation can be obtained at:
% http://www.ctan.org/tex-archive/macros/latex/contrib/sttools/
% Documentation is contained in the stfloats.sty comments as well as in the
% presfull.pdf file. Do not use the stfloats baselinefloat ability as IEEE
% does not allow \baselineskip to stretch. Authors submitting work to the
% IEEE should note that IEEE rarely uses double column equations and
% that authors should try to avoid such use. Do not be tempted to use the
% cuted.sty or midfloat.sty packages (also by Sigitas Tolusis) as IEEE does
% not format its papers in such ways.





% *** PDF, URL AND HYPERLINK PACKAGES ***
%
%\usepackage{url}
% url.sty was written by Donald Arseneau. It provides better support for
% handling and breaking URLs. url.sty is already installed on most LaTeX
% systems. The latest version can be obtained at:
% http://www.ctan.org/tex-archive/macros/latex/contrib/misc/
% Read the url.sty source comments for usage information. Basically,
% \url{my_url_here}.





% *** Do not adjust lengths that control margins, column widths, etc. ***
% *** Do not use packages that alter fonts (such as pslatex).         ***
% There should be no need to do such things with IEEEtran.cls V1.6 and later.
% (Unless specifically asked to do so by the journal or conference you plan
% to submit to, of course. )


% correct bad hyphenation here
\hyphenation{op-tical net-works semi-conduc-tor}


\begin{document}
%  paper title can use linebreaks \\ within to get better formatting as desired
\title{Job Based Architecture: a reduced thread context switching pipeline
concurrency model}


% author names and affiliations
% use a multiple column layout for up to three different
% affiliations
\author{\IEEEauthorblockN{Daniel Sagenschneider}
\IEEEauthorblockA{daniel@officefloor.net}}

% conference papers do not typically use \thanks and this command
% is locked out in conference mode. If really needed, such as for
% the acknowledgment of grants, issue a \IEEEoverridecommandlockouts
% after \documentclass

% for over three affiliations, or if they all won't fit within the width
% of the page, use this alternative format:
% 
%\author{\IEEEauthorblockN{Michael Shell\IEEEauthorrefmark{1},
%Homer Simpson\IEEEauthorrefmark{2},
%James Kirk\IEEEauthorrefmark{3}, 
%Montgomery Scott\IEEEauthorrefmark{3} and
%Eldon Tyrell\IEEEauthorrefmark{4}}
%\IEEEauthorblockA{\IEEEauthorrefmark{1}School of Electrical and Computer Engineering\\
%Georgia Institute of Technology,
%Atlanta, Georgia 30332--0250\\ Email: see http://www.michaelshell.org/contact.html}
%\IEEEauthorblockA{\IEEEauthorrefmark{2}Twentieth Century Fox, Springfield, USA\\
%Email: homer@thesimpsons.com}
%\IEEEauthorblockA{\IEEEauthorrefmark{3}Starfleet Academy, San Francisco, California 96678-2391\\
%Telephone: (800) 555--1212, Fax: (888) 555--1212}
%\IEEEauthorblockA{\IEEEauthorrefmark{4}Tyrell Inc., 123 Replicant Street, Los Angeles, California 90210--4321}}




% use for special paper notices
%\IEEEspecialpapernotice{(Invited Paper)}




% make the title area
\maketitle


\begin{abstract}
By making a thread a parameter to the function, it reduces the thread context
switching of the pipeline concurrency model.  This is accomplished by enabling
the pipeline architecture to behave like a thread-per-request architecture when
there are no performance bottlenecks to isolate.  Discussion is presented on an
initial exploration of this concept with the derived Job Based Architecture. 
The OfficeFloor implementation of the Job Based Architecture is also performance
tested by a request throughput comparison against popular web servers.  The
findings demonstrate OfficeFloor has more consistent throughput performance,
and in many load profiles increased throughput, than popular web servers for
servicing requests for differing characteristics of dynamic content.
\end{abstract}




% For peer review papers, you can put extra information on the cover
% page as needed:
% \ifCLASSOPTIONpeerreview
% \begin{center} \bfseries EDICS Category: 3-BBND \end{center}
% \fi
%
% For peerreview papers, this IEEEtran command inserts a page break and
% creates the second title. It will be ignored for other modes.
\IEEEpeerreviewmaketitle



\section{Introduction}
This paper presents the Job Based Architecture and its OfficeFloor
implementation for reducing the thread context switching of the pipeline
architecture in servicing dynamic content.  As tuning is critical for
performance of web servers~\cite{tuning-important,tuning-os-important}, which I
believe is also important for dynamic content, I find there are trade-offs in
both developing and tuning Internet based applications (e.g. pipelining
incurring high thread context switching or thread-per-request incurring a large
server footprint).  Rather than trying to optimise the choice and tuning of web
server architecture I propose a mechanism for improving the pipeline
architecture by reducing its thread context switching and in appropriate cases
have it behave like a thread-per-request web server.  Furthermore, I will
demonstrate how a thread can be a parameter to a function for isolating
functions containing performance bottlenecks in servicing dynamic content.


\section{Job Based Architecture}
The Job Based Architecture builds on the concepts of the Pipeline
Pattern~\cite{pipeline}, Continuations~\cite{continuations}, Inversion of
Control Pattern~\cite{ioc}, Reactor Pattern~\cite{reactor}, Combine
Pattern~\cite{pipeline} and Wrap Pattern~\cite{pipeline} to use multiple thread
pools to isolate the execution of certain pipelined tasks based on their
dependencies.

Within a Job Based Architecture the application functionality is decomposed into
jobs that are executed sequentially (Pipeline Pattern~\cite{pipeline}).  The
construction of the sequence of jobs is particular to the request being
serviced.  An example sequence of jobs to service a request to retrieve data
from a database is in Table~\ref{tab:example_request_jobs}.

A job is a pipelined task that provides additional meta-data by specifying its
required dependencies and through continuations allows dynamic construction of
the pipeline.  Each job is constructed via the Inversion of Control Pattern and
through extrinsic dependency management~\cite{ioc} the job defines its required
dependencies (e.g. database connection).  Based on the dependencies required of
the job, the job will be dispatched (Reactor Pattern~\cite{reactor}) to a
particular thread pool (Combine Pattern~\cite{pipeline}) for execution.
The Wrap Pattern~\cite{pipeline} is used within the thread pools to handle
bursts of requests.  The dynamic construction of the sequence of jobs is
achieved by providing each job with a list of possible
continuations~\cite{continuations} that encapsulate the construction and
dispatching of the next possible jobs in the sequence. Managing state between
jobs for the application is contained within the dependencies.  For the sequence
of jobs in Table~\ref{tab:example_request_jobs}, thread pools are configured as
per Table~\ref{tab:example_request_thread_pools} to execute the jobs.

Thread context switching is reduced in the Job Based Architecture over more
conventional pipeline architectures.  As the job dispatching to thread pools has
the dependency information of the next job, it can determine if the current
thread may execute the next job and subsequently reduce the thread context
switching.  In Table~\ref{tab:example_request_thread_pools}, the Default thread
pool does not contain threads but executes its jobs by re-using the thread of
the previous job.  The result is that the Parse HTTP request, Dispatch HTTP
request and Validate client data jobs are executed by the thread from the
Network thread pool, and the Render HTTP response and Write HTTP response jobs
are executed by the thread from the Database thread pool.

The intention of the Job Based Architecture is to re-use the current thread to
execute the next job in the pipeline if the next job does not have a performance
bottleneck to isolate.  The resulting behaviour when there are no performance
bottlenecks is similar to a thread-per-request architecture by servicing the
request with one thread.  This reduces the thread context switching typically
incurred between stages of the pipeline architecture.

As extrinsic dependency management is available at compile time regarding the
jobs, compile time optimisations may be used to reduce thread decision
overheads.  This also enables providing tools to aid developers such as
OfficeFloor's graphical configuration editors.  Furthermore,  I expect over the
evolution of the application the changing performance bottlenecks within tasks
can be better identified through task dependencies than task type.

\begin{table}[!t]
\renewcommand{\arraystretch}{1.3}
\caption{Example jobs and their dependencies}
\label{tab:example_request_jobs}
\centering
\begin{tabular}{l||l}
\hline
\bfseries Job & \bfseries Dependencies \\
\hline\hline
Read data from socket & Selector, Socket \\
\hline
Parse HTTP request & Data read \\
\hline
Dispatch HTTP request & HTTP request \\
\hline
Validate client data & HTTP request \\
\hline
Retrieve data from database & Client data, Database connection \\
\hline
Render HTTP response & Database data \\
\hline
Write HTTP response & HTTP response, Socket \\
\hline
\end{tabular}
\end{table}

\begin{table}[!t]
\renewcommand{\arraystretch}{1.3}
\caption{Example assigning of thread pool to jobs by dependency}
\label{tab:example_request_thread_pools}
\centering
\begin{tabular}{l||l||l}
\hline
\bfseries Thread Pool & \bfseries Dependency & \bfseries Jobs \\
\hline\hline
Network & Selector & Read data from socket \\
\hline
Database & Database connection & Retrieve data from database \\
\hline
Default & - & Parse HTTP request, \\
& & Dispatch HTTP request, \\
& & Validate client data, \\ 
& & Render HTTP response, \\
& & Write HTTP response \\
\hline
\end{tabular}
\end{table}

\section{Thread is a Parameter to the Function}
While the previous section explains the mechanism, I propose more fundamentally
that a thread should be a parameter to the function much like a continuation.
The function has a dependency on a thread to be executed and the thread used by
the function is determined as a transitive dependency of the function's
dependencies.  Should the thread not be derived as a transitive dependency, the
previous function's thread becomes the implicit thread to use.  This is similar
to an implicit continuation~\cite{continuations} that executes the next function
after a function returns.  Therefore, a thread is a dependency of the function
and as such a parameter.

Furthermore, I propose a more complete inversion of control solution is achieved
by making the executing thread and the possible continuations of the function be
parameters of the function.  The function has dependencies as inputs (with one
being the thread to use) and continuations for where execution is to continue
(possibly by another thread) with the outputs of the function.

The resulting concurrency model is similar to a pipeline of work managed within
an office where people are assigned to undertake a particular function.
The office process is formed by connecting tasks together with continuations
(e.g. outboxes which feed as input to the next possible tasks in the process).
The tasks are grouped based on similarity (e.g. dependency on particular skill
or resource) and based on that similarity have a particular team (thread pool)
assigned.  Teams may take on smaller tasks in the process before handing off to
another team for improved efficiency.  Differing teams can then be scaled up and
down based on the load of tasks in the groupings for which they are responsible.
The resulting model ensures one function does not consume all the people
(threads).  This is the original inspiration behind OfficeFloor and where I
derived its name.



\section{Performance Evaluation}
To focus evaluation on servicing requests with and without bottlenecks,
connections sending only one of the following requests is established with the
load profile of connections varied:
\begin{IEEEitemize}
\item NEWS: requests resources generated from cached content (e.g. browsing the
latest news feeds); and
\item AJAX: requests requiring communication with a database (e.g. AJAX calls to
retrieve/update data in a database).
\end{IEEEitemize}

The AJAX requests depend on the database which is the bottleneck in this
evaluation.  A connection pool of 100 connections is used with a 100 millisecond
sleep in the servicing code to simulate the database interaction.

Each test run is for 60 seconds with each result obtained by averaging 10 runs
after a 5 minute warm up.  Response entities contain only a single byte so there
are no bandwidth issues.  To avoid connection throttling by the web servers, a
sequence of 99 requests are made before disconnecting and reconnecting.

The servers are run on a two 1.4GHz core machine with 3GB RAM while the test
client run on a three 2.2GHz core machine with 6GB RAM.  Server and client are
connected directly via a Cat 6 cross-over cable and run Linux 2.6.32 with full
duplex 1 Gb/s network cards.  File handles and TCP settings are increased to
enable servicing 20,000 concurrent connections.

\subsection{OfficeFloor (O)}
OfficeFloor's web components implicitly constructs threads for servicing
requests and an additional thread pool is configured for jobs with a
\textit{DataSource} dependency.  Figure ~\ref{fig:officefloor_code} contains the
application code and \textit{O} identifies the OfficeFloor results.  All
requests are mapped to the \textit{service} method via URL
continuations~\cite{url-continuation}.


\subsection{Compared Web Severs (L \& M)}
Popular web servers, identified from the Netcraft August 2012 survey, for
servicing requests for dynamic content through PHP~5.3.2 (Apache~2.2.14, Nginx
1.2.3) and Java~1.6.0\_26 (Jetty~8.1.4, Grizzly~2.2.11, Tomcat~7.0.29) are used
for comparison.  IIS is excluded due to requiring a different operating system.
The code is contained in one PHP snippet / Servlet as an \textit{if} statement
to return immediately (NEWS) or sleep on the pooled database connection (AJAX).
As PHP has a share nothing architecture, a compromise of 150 clients is used to
balance the database bottleneck against increased concurrency.  The two best
servers based on throughput are reported.  To focus on evaluating OfficeFloor
and avoid comparisons of these web servers, one of the servers is identified as
\textit{L} (as it limits the number of threads) and the other as \textit{M} (as
it uses many threads - nearly fifty times the number OfficeFloor uses for
the high concurrency loads).



\section{Performance Results}
For the low concurrency where threads are not exhausted
(Fig.~\ref{fig:low_concurrency_throughput}) throughput is reasonably similar.
OfficeFloor~(\textit{O}) does have higher throughput on one NEWS and one AJAX
connection (\textit{1n/1a}) as it only requires two threads to service, while
the other servers require at least three (e.g. one selector thread and two
servicing threads).  As the number of concurrent connections increase, the
latency effects of the network and thread context switching is reduced for all
servers.  As \textit{L} does not have job management overheads it has lower
servicing latency and subsequently more throughput.

\begin{figure}[!t]
\begin{verbatim}
public class ServiceLogic {
 @FlowInterface
 private static interface Continuations {
  void database();
 }

 public void service(
   ServerHttpConnection conn, 
   Continuations continuations) 
   throws Exception { 
  if (conn.getHttpRequest()
    .getRequestURI().endsWith("N")) {
   conn.getHttpResponse()
    .getEntity().write((byte) 'n');
   return;
  }
  continuations.database();
 }

 public void database(
   ServerHttpConnection conn,
   DataSource dataSource) 
   throws Exception {
  Connection connection = dataSource
   .getConnection();
  Thread.sleep(100);
  connection.close();
  conn.getHttpResponse().getEntity()
   .write((byte) 'a');
 }
}
\end{verbatim}
\caption[Caption for Code]{OfficeFloor application code.\footnotemark }
\label{fig:officefloor_code}
\end{figure}

\footnotetext{Dependencies of the same type are distinguished by qualifying
annotations.}

As the concurrency increases (Fig.~\ref{fig:medium_concurrency_throughput}),
server \textit{L} shows its limitations.  With its lower latency it favours a
load profile of servicing requests quickly.  However, as the database bottleneck
becomes the majority of the load its limited number of threads become blocked
servicing the AJAX requests starving the NEWS requests.  Server \textit{M}
continues to service requests by creating a substantial number of threads, while
OfficeFloor isolates the bottleneck of the AJAX requests to the Database thread
pool leaving the Network threads to service a consistent number of the NEWS
requests.

Increasing concurrency by another magnitude
(Fig.~\ref{fig:high_concurrency_throughput}) causes server \textit{M} to create
a significant number of threads that adversely affects its performance.  While
all servers favour throughput of the quick to service NEWS requests, only
OfficeFloor is able to continue servicing a consistent throughput of both NEWS
and AJAX requests by acting as a pipeline architecture to isolate the database
performance bottleneck of the AJAX request.

\begin{figure}[!t]
\centering 
\includegraphics[width=2.5in]{LowConcurrencyThroughput}
\caption{Low Concurrency Throughput}
\label{fig:low_concurrency_throughput}
\end{figure}


\begin{figure}[!t]
\centering 
\includegraphics[width=2.5in]{MediumConcurrencyThroughput}
\caption{Medium Concurrency Throughput}
\label{fig:medium_concurrency_throughput}
\end{figure}

\begin{figure}[!t]
\centering 
\includegraphics[width=2.5in]{HighConcurrencyThroughput}
\caption{High Concurrency Throughput}
\label{fig:high_concurrency_throughput}
\end{figure}
  


\section{Related Work}
OfficeFloor and its Job Based Architecture is to the best of my knowledge the
first time that the patterns of pipeline, continuations, inversion of control
and reactor have been used together with the abstraction of a thread as a
transitive dependency to achieve a reduced thread context switching pipeline
concurrency model.

Minimising the cost of bottlenecks is one means of improving server performance.
 Monadic threads~\cite{monadic-thread} drive the operating system to provide
asynchronous resources to avoid blocking and starvation.  Comparison work
identified that servers with smaller footprints provide more memory for caching
and subsequently better performance~\cite{low-server-footprint}.  While
minimising the cost of bottlenecks is important to server performance, findings
of this paper also identify that isolation of performance bottlenecks is
important for more consistent server performance.

Various concurrency models have been developed that focus on the task type.
JAWS provides flexible concurrency components to code tasks within~\cite{jaws}.
Aspen combines tasks into modules~\cite{aspen}.  Saburo abstracts the
concurrency model via annotations on the tasks~\cite{saburo}.  Hop enables the
developer to provide code to dynamic decide the concurrency model for tasks at
runtime~\cite{hop}.  I, however, propose the concurrency model should be based
on the thread being a transitive dependency of the task.

Dependencies have been previously used in various models.  The SEDA architecture
uses resource controllers to adapt resource usage of stages for handling
scale~\cite{seda}.  Spring implements dependency injection~\cite{ioc} but it
relies on the Servlet thread-per-request concurrency model.  Capriccio utilises
resource-aware scheduling of operating system level resources~\cite{capriccio}.
Dependency capsules focus on isolation for better availability, however, they
incur thread context switching overheads~\cite{dependency-capsules} avoided by
the Job Based Architecture.  Flux~\cite{flux} provides extrinsic dependency
management information through task inputs and outputs to focus on avoiding
deadlocks by task sequencing and it would likely be beneficial to incorporate
this into OfficeFloor in future work.

The Actor Model provides the mechanism to isolate bottlenecks, however, the
decoupling of message from sender~\cite{actors} allowing it to be scalable
across distributed nodes is losing local efficiencies by not working across
organisational (iOrg) boundaries for the smaller tasks.


\section{Conclusion and Future Work}
Reducing the thread context switching of the pipeline architecture is achieved
by the thread being a transitive dependency of the task.  This is shown to
provide more consistent performance across differing load profiles than popular
web servers for servicing dynamic content.  Furthermore, in many load profiles
it provides improved throughput performance.

Future work will look at whether the more consistent throughput performance will
improve servicing load profiles of real world traffic.  Ease of development will
be explored along with confirming a relationship between dependency and
performance bottleneck.  Admission controls that are tailored to the dependency
of the bottleneck will be incorporated.  Comparisons against other architectures
will be undertaken, such as event-based architectures to compare developer
customisations of concurrency against OfficeFloor's transitive thread dependency
concurrency model.  Expanding to other Internet services is another future area.

Though not discussed in this paper, OfficeFloor implements process
continuations~\cite{process-continuation} to provide parallel processing.  I
predict with further research this will prove useful to address performance
bottleneck problems of interacting with multiple downstream systems, such as the
reverse 10K problem~\cite{reverse-ten-k-problem}.







% conference papers do not normally have an appendix


% use section* for acknowledgement
\section*{Acknowledgment} I thank my wife Melanie for her patience and support
in my exploration of the Job Based Architecture through developing OfficeFloor. 
I also thank my friend Matthew Brown for being a sounding board to many of my
ideas.



% An example of a floating figure using the graphicx package.
% Note that \label must occur AFTER (or within) \caption.
% For figures, \caption should occur after the \includegraphics.
% Note that IEEEtran v1.7 and later has special internal code that
% is designed to preserve the operation of \label within \caption
% even when the captionsoff option is in effect. However, because
% of issues like this, it may be the safest practice to put all your
% \label just after \caption rather than within \caption{}.
%
% Reminder: the "draftcls" or "draftclsnofoot", not "draft", class
% option should be used if it is desired that the figures are to be
% displayed while in draft mode.
%
%\begin{figure}[!t]
%\centering
%\includegraphics[width=2.5in]{myfigure}
% where an .eps filename suffix will be assumed under latex, 
% and a .pdf suffix will be assumed for pdflatex; or what has been declared
% via \DeclareGraphicsExtensions.
%\caption{Simulation Results}
%\label{fig_sim}
%\end{figure}

% Note that IEEE typically puts floats only at the top, even when this
% results in a large percentage of a column being occupied by floats.


% An example of a double column floating figure using two subfigures.
% (The subfig.sty package must be loaded for this to work.)
% The subfigure \label commands are set within each subfloat command, the
% \label for the overall figure must come after \caption.
% \hfil must be used as a separator to get equal spacing.
% The subfigure.sty package works much the same way, except \subfigure is
% used instead of \subfloat.
%
%\begin{figure*}[!t]
%\centerline{\subfloat[Case I]\includegraphics[width=2.5in]{subfigcase1}%
%\label{fig_first_case}}
%\hfil
%\subfloat[Case II]{\includegraphics[width=2.5in]{subfigcase2}%
%\label{fig_second_case}}}
%\caption{Simulation results}
%\label{fig_sim}
%\end{figure*}
%
% Note that often IEEE papers with subfigures do not employ subfigure
% captions (using the optional argument to \subfloat), but instead will
% reference/describe all of them (a), (b), etc., within the main caption.


% An example of a floating table. Note that, for IEEE style tables, the 
% \caption command should come BEFORE the table. Table text will default to
% \footnotesize as IEEE normally uses this smaller font for tables.
% The \label must come after \caption as always.
%
%\begin{table}[!t]
%% increase table row spacing, adjust to taste
%\renewcommand{\arraystretch}{1.3}
% if using array.sty, it might be a good idea to tweak the value of
% \extrarowheight as needed to properly center the text within the cells
%\caption{An Example of a Table}
%\label{table_example}
%\centering
%% Some packages, such as MDW tools, offer better commands for making tables
%% than the plain LaTeX2e tabular which is used here.
%\begin{tabular}{|c||c|}
%\hline
%One & Two\\
%\hline
%Three & Four\\
%\hline
%\end{tabular}
%\end{table}


% Note that IEEE does not put floats in the very first column - or typically
% anywhere on the first page for that matter. Also, in-text middle ("here")
% positioning is not used. Most IEEE journals/conferences use top floats
% exclusively. Note that, LaTeX2e, unlike IEEE journals/conferences, places
% footnotes above bottom floats. This can be corrected via the \fnbelowfloat
% command of the stfloats package.




% trigger a \newpage just before the given reference
% number - used to balance the columns on the last page
% adjust value as needed - may need to be readjusted if
% the document is modified later
%\IEEEtriggeratref{8}
% The "triggered" command can be changed if desired:
%\IEEEtriggercmd{\enlargethispage{-5in}}

% references section

% can use a bibliography generated by BibTeX as a .bbl file
% BibTeX documentation can be easily obtained at:
% http://www.ctan.org/tex-archive/biblio/bibtex/contrib/doc/
% The IEEEtran BibTeX style support page is at:
% http://www.michaelshell.org/tex/ieeetran/bibtex/
\bibliographystyle{ieee/IEEEtran}
% argument is your BibTeX string definitions and bibliography database(s)
\bibliography{jba}


% that's all folks
\end{document}


