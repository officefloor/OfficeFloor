
%% bare_conf.tex
%% V1.3
%% 2007/01/11
%% by Michael Shell
%% See:
%% http://www.michaelshell.org/
%% for current contact information.
%%
%% This is a skeleton file demonstrating the use of IEEEtran.cls
%% (requires IEEEtran.cls version 1.7 or later) with an IEEE conference paper.
%%
%% Support sites:
%% http://www.michaelshell.org/tex/ieeetran/
%% http://www.ctan.org/tex-archive/macros/latex/contrib/IEEEtran/
%% and
%% http://www.ieee.org/

%%*************************************************************************
%% Legal Notice:
%% This code is offered as-is without any warranty either expressed or
%% implied; without even the implied warranty of MERCHANTABILITY or
%% FITNESS FOR A PARTICULAR PURPOSE! 
%% User assumes all risk.
%% In no event shall IEEE or any contributor to this code be liable for
%% any damages or losses, including, but not limited to, incidental,
%% consequential, or any other damages, resulting from the use or misuse
%% of any information contained here.
%%
%% All comments are the opinions of their respective authors and are not
%% necessarily endorsed by the IEEE.
%%
%% This work is distributed under the LaTeX Project Public License (LPPL)
%% ( http://www.latex-project.org/ ) version 1.3, and may be freely used,
%% distributed and modified. A copy of the LPPL, version 1.3, is included
%% in the base LaTeX documentation of all distributions of LaTeX released
%% 2003/12/01 or later.
%% Retain all contribution notices and credits.
%% ** Modified files should be clearly indicated as such, including  **
%% ** renaming them and changing author support contact information. **
%%
%% File list of work: IEEEtran.cls, IEEEtran_HOWTO.pdf, bare_adv.tex,
%%                    bare_conf.tex, bare_jrnl.tex, bare_jrnl_compsoc.tex
%%*************************************************************************

% *** Authors should verify (and, if needed, correct) their LaTeX system  ***
% *** with the testflow diagnostic prior to trusting their LaTeX platform ***
% *** with production work. IEEE's font choices can trigger bugs that do  ***
% *** not appear when using other class files.                            ***
% The testflow support page is at:
% http://www.michaelshell.org/tex/testflow/



% Note that the a4paper option is mainly intended so that authors in
% countries using A4 can easily print to A4 and see how their papers will
% look in print - the typesetting of the document will not typically be
% affected with changes in paper size (but the bottom and side margins will).
% Use the testflow package mentioned above to verify correct handling of
% both paper sizes by the user's LaTeX system.
%
% Also note that the "draftcls" or "draftclsnofoot", not "draft", option
% should be used if it is desired that the figures are to be displayed in
% draft mode.
%
\documentclass[conference]{ieee/IEEEtran}
% Add the compsoc option for Computer Society conferences.
%
% If IEEEtran.cls has not been installed into the LaTeX system files,
% manually specify the path to it like:
% \documentclass[conference]{../sty/IEEEtran}

 



% Some very useful LaTeX packages include:
% (uncomment the ones you want to load)


% *** MISC UTILITY PACKAGES ***
%
%\usepackage{ifpdf}
% Heiko Oberdiek's ifpdf.sty is very useful if you need conditional
% compilation based on whether the output is pdf or dvi.
% usage:
% \ifpdf
%   % pdf code
% \else
%   % dvi code
% \fi
% The latest version of ifpdf.sty can be obtained from:
% http://www.ctan.org/tex-archive/macros/latex/contrib/oberdiek/
% Also, note that IEEEtran.cls V1.7 and later provides a builtin
% \ifCLASSINFOpdf conditional that works the same way.
% When switching from latex to pdflatex and vice-versa, the compiler may
% have to be run twice to clear warning/error messages.






% *** CITATION PACKAGES ***
%
\usepackage{cite}
% cite.sty was written by Donald Arseneau
% V1.6 and later of IEEEtran pre-defines the format of the cite.sty package
% \cite{} output to follow that of IEEE. Loading the cite package will
% result in citation numbers being automatically sorted and properly
% "compressed/ranged". e.g., [1], [9], [2], [7], [5], [6] without using
% cite.sty will become [1], [2], [5]--[7], [9] using cite.sty. cite.sty's
% \cite will automatically add leading space, if needed. Use cite.sty's
% noadjust option (cite.sty V3.8 and later) if you want to turn this off.
% cite.sty is already installed on most LaTeX systems. Be sure and use
% version 4.0 (2003-05-27) and later if using hyperref.sty. cite.sty does
% not currently provide for hyperlinked citations.
% The latest version can be obtained at:
% http://www.ctan.org/tex-archive/macros/latex/contrib/cite/
% The documentation is contained in the cite.sty file itself.






% *** GRAPHICS RELATED PACKAGES ***
%
\ifCLASSINFOpdf
   \usepackage[pdftex]{graphicx}
  % declare the path(s) where your graphic files are
   \graphicspath{{./pdf/}}
  % and their extensions so you won't have to specify these with
  % every instance of \includegraphics
   \DeclareGraphicsExtensions{.pdf}
\else
  % or other class option (dvipsone, dvipdf, if not using dvips). graphicx
  % will default to the driver specified in the system graphics.cfg if no
  % driver is specified.
  % \usepackage[dvips]{graphicx}
  % declare the path(s) where your graphic files are
  % \graphicspath{{../eps/}}
  % and their extensions so you won't have to specify these with
  % every instance of \includegraphics
  % \DeclareGraphicsExtensions{.eps}
\fi
% graphicx was written by David Carlisle and Sebastian Rahtz. It is
% required if you want graphics, photos, etc. graphicx.sty is already
% installed on most LaTeX systems. The latest version and documentation can
% be obtained at: 
% http://www.ctan.org/tex-archive/macros/latex/required/graphics/
% Another good source of documentation is "Using Imported Graphics in
% LaTeX2e" by Keith Reckdahl which can be found as epslatex.ps or
% epslatex.pdf at: http://www.ctan.org/tex-archive/info/
%
% latex, and pdflatex in dvi mode, support graphics in encapsulated
% postscript (.eps) format. pdflatex in pdf mode supports graphics
% in .pdf, .jpeg, .png and .mps (metapost) formats. Users should ensure
% that all non-photo figures use a vector format (.eps, .pdf, .mps) and
% not a bitmapped formats (.jpeg, .png). IEEE frowns on bitmapped formats
% which can result in "jaggedy"/blurry rendering of lines and letters as
% well as large increases in file sizes.
%
% You can find documentation about the pdfTeX application at:
% http://www.tug.org/applications/pdftex





% *** MATH PACKAGES ***
%
%\usepackage[cmex10]{amsmath}
% A popular package from the American Mathematical Society that provides
% many useful and powerful commands for dealing with mathematics. If using
% it, be sure to load this package with the cmex10 option to ensure that
% only type 1 fonts will utilized at all point sizes. Without this option,
% it is possible that some math symbols, particularly those within
% footnotes, will be rendered in bitmap form which will result in a
% document that can not be IEEE Xplore compliant!
%
% Also, note that the amsmath package sets \interdisplaylinepenalty to 10000
% thus preventing page breaks from occurring within multiline equations. Use:
%\interdisplaylinepenalty=2500
% after loading amsmath to restore such page breaks as IEEEtran.cls normally
% does. amsmath.sty is already installed on most LaTeX systems. The latest
% version and documentation can be obtained at:
% http://www.ctan.org/tex-archive/macros/latex/required/amslatex/math/





% *** SPECIALIZED LIST PACKAGES ***
%
%\usepackage{algorithmic}
% algorithmic.sty was written by Peter Williams and Rogerio Brito.
% This package provides an algorithmic environment fo describing algorithms.
% You can use the algorithmic environment in-text or within a figure
% environment to provide for a floating algorithm. Do NOT use the algorithm
% floating environment provided by algorithm.sty (by the same authors) or
% algorithm2e.sty (by Christophe Fiorio) as IEEE does not use dedicated
% algorithm float types and packages that provide these will not provide
% correct IEEE style captions. The latest version and documentation of
% algorithmic.sty can be obtained at:
% http://www.ctan.org/tex-archive/macros/latex/contrib/algorithms/
% There is also a support site at:
% http://algorithms.berlios.de/index.html
% Also of interest may be the (relatively newer and more customizable)
% algorithmicx.sty package by Szasz Janos:
% http://www.ctan.org/tex-archive/macros/latex/contrib/algorithmicx/




% *** ALIGNMENT PACKAGES ***
%
\usepackage{array}
% Frank Mittelbach's and David Carlisle's array.sty patches and improves
% the standard LaTeX2e array and tabular environments to provide better
% appearance and additional user controls. As the default LaTeX2e table
% generation code is lacking to the point of almost being broken with
% respect to the quality of the end results, all users are strongly
% advised to use an enhanced (at the very least that provided by array.sty)
% set of table tools. array.sty is already installed on most systems. The
% latest version and documentation can be obtained at:
% http://www.ctan.org/tex-archive/macros/latex/required/tools/


%\usepackage{mdwmath}
%\usepackage{mdwtab}
% Also highly recommended is Mark Wooding's extremely powerful MDW tools,
% especially mdwmath.sty and mdwtab.sty which are used to format equations
% and tables, respectively. The MDWtools set is already installed on most
% LaTeX systems. The lastest version and documentation is available at:
% http://www.ctan.org/tex-archive/macros/latex/contrib/mdwtools/


% IEEEtran contains the IEEEeqnarray family of commands that can be used to
% generate multiline equations as well as matrices, tables, etc., of high
% quality.


%\usepackage{eqparbox}
% Also of notable interest is Scott Pakin's eqparbox package for creating
% (automatically sized) equal width boxes - aka "natural width parboxes".
% Available at:
% http://www.ctan.org/tex-archive/macros/latex/contrib/eqparbox/





% *** SUBFIGURE PACKAGES ***
%\usepackage[tight,footnotesize]{subfigure}
% subfigure.sty was written by Steven Douglas Cochran. This package makes it
% easy to put subfigures in your figures. e.g., "Figure 1a and 1b". For IEEE
% work, it is a good idea to load it with the tight package option to reduce
% the amount of white space around the subfigures. subfigure.sty is already
% installed on most LaTeX systems. The latest version and documentation can
% be obtained at:
% http://www.ctan.org/tex-archive/obsolete/macros/latex/contrib/subfigure/
% subfigure.sty has been superceeded by subfig.sty.



%\usepackage[caption=false]{caption}
%\usepackage[font=footnotesize]{subfig}
% subfig.sty, also written by Steven Douglas Cochran, is the modern
% replacement for subfigure.sty. However, subfig.sty requires and
% automatically loads Axel Sommerfeldt's caption.sty which will override
% IEEEtran.cls handling of captions and this will result in nonIEEE style
% figure/table captions. To prevent this problem, be sure and preload
% caption.sty with its "caption=false" package option. This is will preserve
% IEEEtran.cls handing of captions. Version 1.3 (2005/06/28) and later 
% (recommended due to many improvements over 1.2) of subfig.sty supports
% the caption=false option directly:
%\usepackage[caption=false,font=footnotesize]{subfig}
%
% The latest version and documentation can be obtained at:
% http://www.ctan.org/tex-archive/macros/latex/contrib/subfig/
% The latest version and documentation of caption.sty can be obtained at:
% http://www.ctan.org/tex-archive/macros/latex/contrib/caption/




% *** FLOAT PACKAGES ***
%
%\usepackage{fixltx2e}
% fixltx2e, the successor to the earlier fix2col.sty, was written by
% Frank Mittelbach and David Carlisle. This package corrects a few problems
% in the LaTeX2e kernel, the most notable of which is that in current
% LaTeX2e releases, the ordering of single and double column floats is not
% guaranteed to be preserved. Thus, an unpatched LaTeX2e can allow a
% single column figure to be placed prior to an earlier double column
% figure. The latest version and documentation can be found at:
% http://www.ctan.org/tex-archive/macros/latex/base/



%\usepackage{stfloats}
% stfloats.sty was written by Sigitas Tolusis. This package gives LaTeX2e
% the ability to do double column floats at the bottom of the page as well
% as the top. (e.g., "\begin{figure*}[!b]" is not normally possible in
% LaTeX2e). It also provides a command:
%\fnbelowfloat
% to enable the placement of footnotes below bottom floats (the standard
% LaTeX2e kernel puts them above bottom floats). This is an invasive package
% which rewrites many portions of the LaTeX2e float routines. It may not work
% with other packages that modify the LaTeX2e float routines. The latest
% version and documentation can be obtained at:
% http://www.ctan.org/tex-archive/macros/latex/contrib/sttools/
% Documentation is contained in the stfloats.sty comments as well as in the
% presfull.pdf file. Do not use the stfloats baselinefloat ability as IEEE
% does not allow \baselineskip to stretch. Authors submitting work to the
% IEEE should note that IEEE rarely uses double column equations and
% that authors should try to avoid such use. Do not be tempted to use the
% cuted.sty or midfloat.sty packages (also by Sigitas Tolusis) as IEEE does
% not format its papers in such ways.





% *** PDF, URL AND HYPERLINK PACKAGES ***
%
%\usepackage{url}
% url.sty was written by Donald Arseneau. It provides better support for
% handling and breaking URLs. url.sty is already installed on most LaTeX
% systems. The latest version can be obtained at:
% http://www.ctan.org/tex-archive/macros/latex/contrib/misc/
% Read the url.sty source comments for usage information. Basically,
% \url{my_url_here}.





% *** Do not adjust lengths that control margins, column widths, etc. ***
% *** Do not use packages that alter fonts (such as pslatex).         ***
% There should be no need to do such things with IEEEtran.cls V1.6 and later.
% (Unless specifically asked to do so by the journal or conference you plan
% to submit to, of course. )


% correct bad hyphenation here
\hyphenation{op-tical net-works semi-conduc-tor}


\begin{document}
%
% paper title
% can use linebreaks \\ within to get better formatting as desired
\title{Job Based Architecture: isolating bottlenecks by \\
awareness of task characteristics via dependencies}


% author names and affiliations
% use a multiple column layout for up to three different
% affiliations
\author{\IEEEauthorblockN{Daniel Sagenschneider}
\IEEEauthorblockA{Email: daniel@officefloor.net}}

% conference papers do not typically use \thanks and this command
% is locked out in conference mode. If really needed, such as for
% the acknowledgment of grants, issue a \IEEEoverridecommandlockouts
% after \documentclass

% for over three affiliations, or if they all won't fit within the width
% of the page, use this alternative format:
% 
%\author{\IEEEauthorblockN{Michael Shell\IEEEauthorrefmark{1},
%Homer Simpson\IEEEauthorrefmark{2},
%James Kirk\IEEEauthorrefmark{3}, 
%Montgomery Scott\IEEEauthorrefmark{3} and
%Eldon Tyrell\IEEEauthorrefmark{4}}
%\IEEEauthorblockA{\IEEEauthorrefmark{1}School of Electrical and Computer Engineering\\
%Georgia Institute of Technology,
%Atlanta, Georgia 30332--0250\\ Email: see http://www.michaelshell.org/contact.html}
%\IEEEauthorblockA{\IEEEauthorrefmark{2}Twentieth Century Fox, Springfield, USA\\
%Email: homer@thesimpsons.com}
%\IEEEauthorblockA{\IEEEauthorrefmark{3}Starfleet Academy, San Francisco, California 96678-2391\\
%Telephone: (800) 555--1212, Fax: (888) 555--1212}
%\IEEEauthorblockA{\IEEEauthorrefmark{4}Tyrell Inc., 123 Replicant Street, Los Angeles, California 90210--4321}}




% use for special paper notices
%\IEEEspecialpapernotice{(Invited Paper)}




% make the title area
\maketitle


\begin{abstract}
By making a thread a parameter to the function, with the thread determined as a
transitive dependency of the function's dependencies, it enables isolating
perfomance bottlenecks of a function to not consume all threads and
subsequently starve other unrelated functions of a thread.  This paper provides
discussion and findings on an initial exploration of this concept with the
derived Job Based Architecture and tests the feasibility by comparison of the
OfficeFloor implementation against other popular Web Servers.  The findings
demonstrate that overheads involved of making a thread a parameter to a
function do not significantly detriment performance and at a high number of
concurrent requests provides more consistent performance over
thread-per-request Web Servers for servicing differing characteristics of
dynamic HTTP requests.
\end{abstract}




% For peer review papers, you can put extra information on the cover
% page as needed:
% \ifCLASSOPTIONpeerreview
% \begin{center} \bfseries EDICS Category: 3-BBND \end{center}
% \fi
%
% For peerreview papers, this IEEEtran command inserts a page break and
% creates the second title. It will be ignored for other modes.
\IEEEpeerreviewmaketitle



\section{Introduction}
This paper presents OfficeFloor and its underlying Job Based Architecture to
enable isolating performance bottlenecks rather than relying on server tuning to
achieve more consistent performance. As performance tuning is necessary for
existing Web Server architectures \cite{tuning-important} along with maintaining
a small server footprint \cite{low-server-footprint}, I find there are
performance trade-offs that require significant involvement by developers in
both developing and tuning Web based applications (e.g. increasing number of
threads for increased concurrency but then reduce number of threads for smaller
server footprint).  Rather than trying to optimise the tuning of the Web Server
I propose that isolating performance bottlenecks to better decouple their
affects will provide more consistent performance.  Furthermore, through the
explanation of the Job Based Architecture I will demonstrate how a thread can be
a parameter to a function.


\section{Job Based Architecture}
The Job Based Architecture builds on the concepts of the Pipeline
Pattern~\cite{pipeline}, Continuations~\cite{continuations}, Reactor
Pattern~\cite{reactor}, Inversion of Control Pattern~\cite{ioc}, Combine
Pattern~\cite{pipeline} and Wrap Pattern~\cite{pipeline} to enable utilising
multiple thread pools to isolate processing of certain pipelined tasks based on
their characteristics.

Within a Job Based Architecture the application functionality is decomposed
into jobs that are executed sequentially (Pipeline pattern~\cite{pipeline}).  The
construction of the sequence of jobs is dynamic and particular to the HTTP
request being serviced.  An example sequence of jobs to service a HTTP request
to dynamically retrieve data in a database would be as table~\ref{tab:example_request_jobs}. 

\begin{table}[!t]
\renewcommand{\arraystretch}{1.3}
\caption{Example jobs to service a dynamic HTTP request}
\label{tab:example_request_jobs}
\centering
\begin{tabular}{l||l}
\hline
\bfseries Job & \bfseries Dependency \\
\hline
\hline
Read data from Socket & Selector, Socket \\
\hline
Parse HTTP Request & Data read \\
\hline
Dispatch HTTP Request & HTTP request \\
\hline
Validate client data & HTTP request \\
\hline
Retrieve Data from database & Client data, Database connection \\
\hline
Render HTTP Response & Database data \\
\hline
Write HTTP Response & HTTP Response, Socket \\
\hline
\end{tabular}
\end{table}

A job provides additional meta-data regarding its processing characteristics
over a pipelined task by specifying its required dependencies and through
continuations allows dynamic construction of the job sequence.  Each job is
constructed via the Inversion of Control pattern and through extrinsic
dependency management \cite{ioc} the job defines its required dependencies (e.g.
Database Connection).  Based on the dependencies required of the job, the job
will be dispatched (Reactor Pattern~\cite{reactor}) to a particular thread pool
(Combine Pattern~\cite{pipeline}) for execution. Handling bursts in HTTP
requests is achieved by utilising the Wrap Pattern~\cite{pipeline} within the
thread pools.  The dynamic construction of the sequence of jobs is achieved by
providing each job a list of possible continuations \cite{continuations} that
encapsulate the construction and dispatching of the possible next jobs in the
sequence. Managing state between jobs for the application is contained within
the dependencies.  For the sequence of jobs in
table~\ref{tab:example_request_jobs}, thread pools are configured as
table~\ref{tab:example_request_thread_pools} to execute the jobs.

\begin{table}[!t]
\renewcommand{\arraystretch}{1.3}
\caption{Example assigning of thread pool to jobs by dependency}
\label{tab:example_request_thread_pools}
\centering
\begin{tabular}{l||l||l}
\hline
\bfseries Thread Pool & \bfseries Dependency & \bfseries Jobs \\
\hline
\hline
Network & Selector & Read data from Socket \\
\hline
Database & Database Connection & Retrieve Data from database \\
\hline
Default & - & Parse HTTP Request, \\
& & Dispatch HTTP Request, \\
& & Validate client data, \\ 
& & Render HTTP Response, \\
& & Write HTTP Response \\
\hline
\end{tabular}
\end{table}

Thread context switching and memory management overheads due to queuing jobs is
reduced in the Job Based Architecture, as the job dispatching to thread pools
encapsulated in the continuation utilises the dependencies to determine if the
next job may be executed by the current thread.  In
table~\ref{tab:example_request_thread_pools}, the \textit{Default} thread pool
does not actually contain threads but executes its jobs by re-using the thread
of the previous job.  In other words, the Parse HTTP Request, Dispatch HTTP
Request and Validate client data jobs are executed by the thread from the
Network thread pool and the Render HTTP Response and Write HTTP Response are
executed by the thread from the Database thread pool.  Write HTTP Response may
however be completed by the Network thread pool if the send socket buffer is
full.

The resulting intention of the Job Based Architecture is that bottlenecks in
servicing a particular stage of the dynamic HTTP request (e.g. blocking network
I/O) is assigned to a particular thread pool and does not affect other stages of
processing the dynamic HTTP requests not subject to the bottleneck (e.g.
blocking disk I/O). Furthermore, should there be no performance bottlenecks to
isolate (e.g. database data is cached) the execution of the HTTP request may be
executed in entirity by one thread, allowing the Job Based Architecture to
behave like an event-based architecture in this case.

Also as extrinsic dependency management is available at compile time, compile
time optimisations may be utilised to reduce overheads.  Furthermore, as the Job
Based Architecture uses job dependencies to determine the appropriate thread
pool for executing each job it reduces arrising bottlenecks from human
configuration/coding errors based on job type especially as jobs may change
their dependencies over the evolution of the application.


\section{Thread is a Parameter to the Function}
While the previous section discusses the mechanisms of how a Job Based
Architecture is achieved, I am finding more fundamentally that a thread should
be a parameter to the function much like a continuation.  The function has a
dependency on a thread to be executed and the thread \textit{passed} to the
function is determined as a transitive dependency of the function's
dependencies.  Should the thread not be derived as a transitive dependency, the
implicit thread of re-using the previous job's thread is passed (similar to an
implicit continutation that executes the next operation after a function
returns).  And therefore a thread is a dependency for the function and as such a
parameter.

Furthermore providing continuations and a thread as parameters of a function
provides a more complete inversion of control.  The function is provided
dependencies as inputs, with one being the thread to use, and continuations for
where execution is to continue (possibly by another thread) with the outputs of
the function.

The resulting concurrency model is similar to a pipeline of work managed within
an office where people are assigned to undertake a particular function.
The office process is formed by connecting tasks together with continuations
(e.g. "outboxes" which feed as input to the next possible tasks in the process).
 The tasks are grouped based on similarity (e.g. dependency on particular skill
or resource) and based on that similarity have a particular team (thread pool)
assigned; one team is typically not responsible for undertaking all tasks of a
process.  Teams may take on smaller tasks in the process before handing off to
another Team for improved efficiency.  Differing teams can then be scaled up and
down based on the load of tasks for only the groupings they are responsible.
The resulting model disallows one function to consume all the people (threads).
This is the original inspiration behind OfficeFloor and where I derived its
name.



\section{Performance Evaluation}
To focus evaluation on the isolation of a bottleneck, two types of requests were
utilised with differing profiles of number of concurrent connections sending one
of the requests:
\begin{itemize}
\item NEWS: Requests for resources generated from cached content (e.g. browsing
the latest news feeds).
\item AJAX: Requests that each required communication with a database (e.g.
AJAX calls to retrieve/update data in a database).
\end{itemize}

The AJAX requests depend on the database which is the bottleneck in this
evaluation; connection pool of 100 connections with a 100 millisecond sleep in
the servicing code to simulate the database interaction.

Each run was for 60 seconds with each result obtained from averaging 10 runs.
Response entities contained only a single byte to not incur bandwidth issues.
To avoid connection throttling by the Web Servers, a sequence of 99 requests
were made before disconnecting and reconnecting.

The server was run on a two 1.4Ghz core machine with 3GB RAM while the test
client run on a three 2.2Ghz core machine with 6GB RAM.  Server and client were
connected directly via cat6 cross-over cable and ran linux v2.6.32 with 1 Gb
network card.  File handles and TCP settings were increased to enable servicing
20,000 concurrent connections.

\subsection{OfficeFloor (O)}
OfficeFloor's web components implicitly constructs threads for servicing HTTP
requests and an additional thread pool was configured for jobs with a
\textit{DataSource} dependency.  The code is as follows:

\begin{verbatim}
public class ServiceLogic {
 @FlowInterface
 private static interface Continuations {
  void database();
 }

 public void news(
   ServerHttpConnection conn, 
   Continuations continuations) 
   throws Exception { 
  if (conn.getHttpRequest()
    .getRequestURI().endsWith("N")) {
   connection.getHttpResponse()
    .getEntity().write((byte) 'n');
   return;
  }
  continuations.database();
 }

 public void database(
   ServerHttpConnection conn,
   DataSource dataSource) 
   throws Exception {
  Connection connection = dataSource
   .getConnection();
  Thread.sleep(100);
  connection.close();
  conn.getHttpResponse().getEntity()
   .write((byte) 'd');
 }
}
\end{verbatim}
   
OfficeFloor is identified as \textit{O} in the results.  All requests are mapped
to the \textit{news} method via URL continuations \cite{url-continuation}.


\subsection{Compared Web Severs (L \& M)}
Web Servers providing dynamic HTTP request servicing through PHP v5.3.2 (Apache
v2.2.14, Nginx v1.2.3) and Servlets (Jetty v8.1.4, Grizzly v2.2.11, Tomcat
v7.0.29) were used for comparison.  IIS was excluded as it required a different
operating system and was found to be similar in performance profile to another
server \textbf{(TODO run tests on Azure to confirm)}.  The code was contained in
the one PHP snippet / Servlet as an if statement to distinguish between
returning immediately (NEWS) or sleeping on the pooled database connection
(AJAX).  As PHP has a share nothing architecture, a compromise of 150 clients
was used to balance the database bottleneck against increased concurrency.  The
two best servers based on throughput are reported. To focus on evaluating
OfficeFloor and avoid comparisons of these Web Servers, one server will be
identified as \textit{L} (as it limited the number of threads) and the other as
\textit{M} (as it used many threads - nearly fifty times the number OfficeFloor
used for high concurrency).



\section{Performance Results}
For the low concurrency where threads are not exhausted
(figure~\ref{fig:low_concurrency_throughput}), throughput is reasonably similar.
OfficeFloor~(\textit{O}) does have higher throughput on 1 NEWS and 1 AJAX
connection as it only requires two threads to service, while the other servers
requires three (i.e. 1 selector thread and 2 servicing threads).  As the number
of concurrent connections increase, OfficeFloor requires more threads and as
\textit{L} does not incur the job management overheads of the Job Based
Architecture it has lower latency and subsequently more throughput.

As the concurrency increases beyond the number of threads
(figure~\ref{fig:medium_concurrency_throughput}), server \textit{L} shows its
limitations.  With its lower latency it favours a load profile of servicing
requests quickly, however as the database bottleneck becomes the majority of the
load its limited number of threads become blocked servicing the AJAX requests
starving the NEWS requests.  Server \textit{M} continues to service requests by
creating a substantial number of threads, while OfficeFloor isolates the
bottleneck of the AJAX requests to the Database thread pool leaving the Network
threads to service the NEWS requests.

Increasing concurrency by another maginitude
(figure~\ref{fig:high_concurrency_throughput}) causes server \textit{M} to
create a significant number of threads that adversely affects its performance. 
While all servers favour throughput of the NEWS requests, when the load profile
was of a significant number of AJAX requests starvation occurred and only
OfficeFloor by isolating the database performance bottleneck was able to
continue servicing a consistent number of NEWS requests.

\begin{figure}[!t]
\centering 
\includegraphics[width=2.5in]{LowConcurrencyThroughput}
\caption{Low Concurrency Throughput}
\label{fig:low_concurrency_throughput}
\end{figure}


\begin{figure}[!t]
\centering 
\includegraphics[width=2.5in]{MediumConcurrencyThroughput}
\caption{Medium Concurrency Throughput}
\label{fig:medium_concurrency_throughput}
\end{figure}

\begin{figure}[!t]
\centering 
\includegraphics[width=2.5in]{HighConcurrencyThroughput}
\caption{High Concurrency Throughput}
\label{fig:high_concurrency_throughput}
\end{figure}
  


\section{Related Work}
OfficeFloor and its Job Based Architecture is to the best of my knowledge the
first time that the patterns of pipeline, continuations and inversion of control
have been used together along with the abstraction of a thread as a transitive
dependency to achieve a concurrency model that focuses on performance bottleneck
isolation.

Various work and comparison evaluations have identified the benefits of
minimising the affects of bottlenecks.  The SEDA Architecture~\cite{seda}
reduced the level of service and subsequently the cost of bottlenecks at higher
loads.  Monadic threads~\cite{monadic-thread} drive the operating system to
provide asynchronous resources to avoid blocking and starvation.  Comparison
work with multi-threaded pipelined web server architecture demonstrated better
performance than multi-process, multi-thread, SPED and AMPED architectures
\cite{multithread-pipeline}.  Further comparison work identified that servers
with smaller footprints provided more memory for caching and subsequently better
performance \cite{low-server-footprint}.  While minimising the cost of
performance bottlenecks is important to server performance, findings of this
paper also identify that isolation of performance bottlenecks is important to
overall server performance.

Various abstractions of concurrency have been developed that typically focus on
the task type.  JAWS provides modular and flexible concurrency components for
developers to code tasks within \cite{jaws}.  Aspen enables the developer to
combine tasks based on type to modules that utilise adaptive thread allocation
\cite{aspen}.  Saburo abstracts the concurrency model via annotations on the
tasks \cite{saburo}.  Hop enables the developer to provide code to dynamic
decide the concurrency model for tasks at runtime \cite{hop}. \textbf{Actors
something, get reference}.  I however propose the abstraction of the concurrency
model should be based on the thread being a transitive dependency of the task.

Some frameworks/libraries have utilised dependencies in their models.  Spring
implements dependency injection techniques \cite{ioc} however it relies on the
Servlet thread-per-request concurrency model.  Capriccio utilised resource-aware
scheduling of operating system level resources but leaves concurrency to the
developer for application level resources via an API \cite{capriccio}.  Flux
\cite{flux} provide extrinsic dependency management information through task
inputs and outputs but focused on using the information for task sequencing to
avoiding deadlocks rather than bottleneck isolation.



\section{Conclusion and Future Work}
Isolating the performance bottleneck has shown to provide more consistent
performance across differing load profiles by avoiding thread starvation than
the thread-per-request Web Servers for servicing dynamic HTTP requests. 
Furthermore, the overheads in the job management of the Job Based Architecture do not significantly detriment
throughput performance.

Future work will look at whether this more consistent throughput performance
will provide improvements in servicing load profiles of real world traffic.
Further comparisons against other web server architectures will be undertaken,
such as event based-architectures to compare developer customisations of
application concurrency against OfficeFloor's transitive thread dependency
concurrency model.  Also, expanding to other Internet services other than Web
Servers is another area for future work.

Though not discussed in this paper OfficeFloor does implement process
continuations \cite{process-continuation} to provide parallel processing.  I
expect with further research this will prove useful to address performance
bottleneck problems of interacting with multiple downstream systems, such as the
reverse 10K problem \cite{reverse-ten-k-problem}.







% conference papers do not normally have an appendix


% use section* for acknowledgement
\section*{Acknowledgment} I thank my wife Melanie for her patience and support
in my exploration of the Job Based Architecture through developing OfficeFloor. 
I also thank my friend Matthew Brown for being a sounding board to many of my
ideas.



% An example of a floating figure using the graphicx package.
% Note that \label must occur AFTER (or within) \caption.
% For figures, \caption should occur after the \includegraphics.
% Note that IEEEtran v1.7 and later has special internal code that
% is designed to preserve the operation of \label within \caption
% even when the captionsoff option is in effect. However, because
% of issues like this, it may be the safest practice to put all your
% \label just after \caption rather than within \caption{}.
%
% Reminder: the "draftcls" or "draftclsnofoot", not "draft", class
% option should be used if it is desired that the figures are to be
% displayed while in draft mode.
%
%\begin{figure}[!t]
%\centering
%\includegraphics[width=2.5in]{myfigure}
% where an .eps filename suffix will be assumed under latex, 
% and a .pdf suffix will be assumed for pdflatex; or what has been declared
% via \DeclareGraphicsExtensions.
%\caption{Simulation Results}
%\label{fig_sim}
%\end{figure}

% Note that IEEE typically puts floats only at the top, even when this
% results in a large percentage of a column being occupied by floats.


% An example of a double column floating figure using two subfigures.
% (The subfig.sty package must be loaded for this to work.)
% The subfigure \label commands are set within each subfloat command, the
% \label for the overall figure must come after \caption.
% \hfil must be used as a separator to get equal spacing.
% The subfigure.sty package works much the same way, except \subfigure is
% used instead of \subfloat.
%
%\begin{figure*}[!t]
%\centerline{\subfloat[Case I]\includegraphics[width=2.5in]{subfigcase1}%
%\label{fig_first_case}}
%\hfil
%\subfloat[Case II]{\includegraphics[width=2.5in]{subfigcase2}%
%\label{fig_second_case}}}
%\caption{Simulation results}
%\label{fig_sim}
%\end{figure*}
%
% Note that often IEEE papers with subfigures do not employ subfigure
% captions (using the optional argument to \subfloat), but instead will
% reference/describe all of them (a), (b), etc., within the main caption.


% An example of a floating table. Note that, for IEEE style tables, the 
% \caption command should come BEFORE the table. Table text will default to
% \footnotesize as IEEE normally uses this smaller font for tables.
% The \label must come after \caption as always.
%
%\begin{table}[!t]
%% increase table row spacing, adjust to taste
%\renewcommand{\arraystretch}{1.3}
% if using array.sty, it might be a good idea to tweak the value of
% \extrarowheight as needed to properly center the text within the cells
%\caption{An Example of a Table}
%\label{table_example}
%\centering
%% Some packages, such as MDW tools, offer better commands for making tables
%% than the plain LaTeX2e tabular which is used here.
%\begin{tabular}{|c||c|}
%\hline
%One & Two\\
%\hline
%Three & Four\\
%\hline
%\end{tabular}
%\end{table}


% Note that IEEE does not put floats in the very first column - or typically
% anywhere on the first page for that matter. Also, in-text middle ("here")
% positioning is not used. Most IEEE journals/conferences use top floats
% exclusively. Note that, LaTeX2e, unlike IEEE journals/conferences, places
% footnotes above bottom floats. This can be corrected via the \fnbelowfloat
% command of the stfloats package.




% trigger a \newpage just before the given reference
% number - used to balance the columns on the last page
% adjust value as needed - may need to be readjusted if
% the document is modified later
%\IEEEtriggeratref{8}
% The "triggered" command can be changed if desired:
%\IEEEtriggercmd{\enlargethispage{-5in}}

% references section

% can use a bibliography generated by BibTeX as a .bbl file
% BibTeX documentation can be easily obtained at:
% http://www.ctan.org/tex-archive/biblio/bibtex/contrib/doc/
% The IEEEtran BibTeX style support page is at:
% http://www.michaelshell.org/tex/ieeetran/bibtex/
\bibliographystyle{ieee/IEEEtran}
% argument is your BibTeX string definitions and bibliography database(s)
\bibliography{jba}


% that's all folks
\end{document}


